%
% Kurzfassung
%
% @version 1.0
% @author wipatrick
% @created 22. November 2015
% @edited 

\chapter*{Kurzfassung}
\thispagestyle{empty}
Im Rahmen dieser Masterarbeit erfolgt die Konzeption, der Aufbau und die Inbetriebnahme einer technischen Anlage zur Regelung einer Raumtemperatur für den Betrieb mit Modellprädiktiver Regelung. Zunächst werden dazu die Anforderungen analysiert und festgelegt, bevor die Planung der Anlage erfolgt. Nach einer Beschreibung der Installation und deren Funktionsweise wird die Eignung der Anlage zur Raumtemperaturregelung aufgezeigt. Der anschließende Teil dient der Bildung eines Raummodells. Dieses wird schrittweise an die realen Gegebenheiten angepasst und nach einer erfolgten Parameterschätzung evaluiert. Abschließend wird überprüft, ob sich das Modell für den Einsatz mit der Modellprädiktiven Regelung in \textsc{JModelica.org} eignet.

\subsubsection*{Abstract}
In the course of this master thesis a technical system for a room temperature control is designed, installed and put into operation for running with model predictive control. Therefore the requirements are analyzed and defined before the technical system is designed. Afterwards follows an installation instruction and a description of functionality. Subsequently the suitability of the system for the control of room temperature is shown. Following this, the model development takes place. The complexity grows stepwise through an adaption of the model to the real environment. Afterwards a parameter estimation and an evaluation of the model takes place. Finally the suitability of the model for the usage with model predictive control in \textsc{JModelica.org} is verified.\vspace{8\baselineskip}

{\normalsize
\sffett{Schlüsselwortliste}:  Modellprädiktive Regelung, Model Predictive Control, \textsc{JModelica.org}, Optimalsteuerung, Modellbildung, Modelica, Kommunikation technischer Systeme, Modbus RTU, Modbus TCP, Raumtemperaturregelung
}

