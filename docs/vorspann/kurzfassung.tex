%
% Kurzfassung
%
% @version 1.0
% @author wipatrick
% @created 22. November 2015
% @edited 

\chapter*{Kurzfassung}
\thispagestyle{empty}
Im Rahmen dieser Masterarbeit erfolgt die Konzeption, der Aufbau und die Inbetriebnahme einer technischen Anlage zur Regelung einer Raumtemperatur für den Betrieb mit Modellprädiktiver Regelung. Zunächst werden dazu die Anforderungen analysiert und festgelegt, bevor die Planung der Anlage ausgeführt wird. Nach einer Beschreibung der Installation und deren Funktionsweise wird die Eignung der Anlage zur Raumtemperaturregelung aufgezeigt. Der anschließende Teil dient der Bildung eines Raummodells. Dieses wird schrittweise an die realen Gegebenheiten angepasst und nach einer erfolgten Parameterschätzung evaluiert. Abschließend wird überprüft, ob sich das Modell für einen Einsatz mit der Modellprädiktiven Regelung in \textsc{JModelica.org} eignet.

\subsubsection*{Abstract}
In the course of this master thesis a technical system for a space heating control is designed, 
\vspace{8\baselineskip}

{\normalsize
\sffett{Schlüsselwortliste}:  Modellprädiktive Regelung, Model Predictive Control, \textsc{JModelica.org}, \textsc{CasADi}, Modellbildung, Modelica, Kommunikation technischer Systeme, Modbus RTU, Modbus TCP, Raumtemperaturregelung
}

