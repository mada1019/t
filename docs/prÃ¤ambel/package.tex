%
% präambel packages
%
% @version 1.0
% @author wipatrick
% @created 22. November 2015
% @edited  
%
%----------------------------------------------------------------------------------------------------------------------
% Allgemeines
%----------------------------------------------------------------------------------------------------------------------
\usepackage[utf8]{inputenc}
%\DeclareUnicodeCharacter{00A0}{ }
%\DeclareUnicodeCharacter{00A0}{~}
\usepackage[T1]{fontenc}                                         % wichtig für Trennung von Wörtern mit Umlauten
\usepackage{lmodern}
\usepackage{microtype}                                           % verbesserter Randausgleich
\usepackage{url}                                                 % zum Zitieren von url
%\urlstyle{rm}                                                   % Serifenstil für URL im Literaturverzeichnis
%\usepackage[singlelinecheck=false]{caption}                     %justification=RaggedRight, linksbündige 
\usepackage{multirow}
\usepackage{lipsum}
\usepackage{textcomp}                                            % Nummer-Zeichen \textdegree N°
\usepackage{xcolor}                                              % definiert, dass hyperlinks nicht gefärbt sind
\definecolor{black}{gray}{0}                                     % 10% gray
\usepackage[colorlinks=true,linkcolor=black,citecolor=black,urlcolor=black]{hyperref}
\renewcommand*{\theHsection}{\thesection}                        % Anhang richtig verlinkt im InhaltsVZ
\usepackage[ngerman]{babel}                                      % deutsche Trennregeln

%----------------------------------------------------------------------------------------------------------------------
% Radar Chart
%----------------------------------------------------------------------------------------------------------------------
\PassOptionsToPackage{pdf}{pstricks}                             % used for pdflatex
\usepackage{pstricks-add}

\usepackage{dirtree}                                             % Verzeichnisstruktur 
\usepackage{calc}                                                % Listing Nummer Abstand
\usepackage{pifont}                                              % Checkmark

\usepackage{booktabs}
%\usepackage{cite}
\usepackage[square]{natbib}
\setlength{\bibsep}{0.25cm}
\renewcommand{\bibfont}{\small} %\normalsize
\renewcommand*{\bibpreamble}{\interlinepenalty10000\relax}

%----------------------------------------------------------------------------------------------------------------------
% Gänsefüßchen
%----------------------------------------------------------------------------------------------------------------------
\usepackage{xspace}                                              % Gänsefüßchen
\newcommand{\Gun}{\glqq{}}		                                  % Gänsefüßchen unten
\newcommand{\Gob}{\grqq\xspace} 	                                  % Gänsefüßchen oben


%----------------------------------------------------------------------------------------------------------------------
% Layout
%----------------------------------------------------------------------------------------------------------------------
\usepackage[automark]{scrpage2}
\automark[section]{chapter} 
\pagestyle{scrheadings}
\clearscrheadfoot
\ihead{\leftmark}
\ohead{\pagemark}
\renewcommand*{\chapterpagestyle}{scrheadings}
\renewcommand*{\indexpagestyle}{scrheadings}
\usepackage[left=30mm,right=38mm,top=30mm,bottom=20mm]{geometry}
%\addtokomafont{sectioning}{\rmfamily} %Serifenstyle für Section
\renewcommand{\chapterheadstartvskip}{\vspace*{0cm}}
\renewcommand{\chapterheadendvskip}{\vspace*{1.5\baselineskip}}
\setkomafont{chapterprefix}{\large} % Kapitelpräfixgröße
\setkomafont{chapter}{\LARGE} % Kapitelgröße
\setlength{\parskip}{0.5EM} %Paragraphen/Absatzabstand

% Überschriften Serifen-Schriftart 
%\addtokomafont{chapter}{\rmfamily}                               
%\addtokomafont{section}{\rmfamily} 
%\addtokomafont{subsection}{\rmfamily} 
%\addtokomafont{subsubsection}{\rmfamily} 

%----------------------------------------------------------------------------------------------------------------------
% Tabellen- und Abbildungsverzeichnis
%----------------------------------------------------------------------------------------------------------------------
\makeatletter                                                     % Einrücken verhindern
\renewcommand*\l@figure{\@dottedtocline{1}{0em}{2.3em}}           % Default: 1.5em/2.3em
\let\l@table\l@figure
\makeatother

\renewcommand{\thefigure}{Abb. \arabic{chapter}.\arabic{figure}} 
\renewcommand{\thetable}{Tab. \arabic{chapter}.\arabic{table}}
\renewcommand{\theequation}{Gl. \arabic{equation}}
\renewcommand*{\figureformat}{\thefigure} 
\renewcommand*{\tableformat}{\thetable} 
\renewcommand*{\captionformat}{: }  

\addtokomafont{caption}{\small\sffamily}		                   % Bild-/Tabellenunterschriften klein & serifenlos
\addtokomafont{captionlabel}{\sffamily\bfseries}                  % Bild-/Tabelllabel fett & serifenlos

\AtBeginDocument{                                                 % Doppelpunkt nach Nummern 
  \addtocontents{lof}{\protect\def\protect\autodot{:}}% 
  \addtocontents{lot}{\protect\def\protect\autodot{:}}% 
} 

%----------------------------------------------------------------------------------------------------------------------
% Tabellen
%----------------------------------------------------------------------------------------------------------------------
\usepackage{tabularx}
\usepackage{stypackage/stackengine}                                % Pfeile in Tabelle
\usepackage{varwidth}
\usepackage{longtable}                                             % Ermöglicht Tabellen über Seitenumbruch
\usepackage[]{threeparttable}                                      % Fußnoten in Tabellen
\usepackage{booktabs}
\usepackage{pdflscape}
\usepackage{colortbl}
\usepackage{enumitem}                                              % Liste in Tabelle
\usepackage{arydshln}                                              % Dashedlines in Tables
\usepackage{hhline}

\newcommand\Thickvrule[1]{
  \multicolumn{1}{!{\vrule width 1.5pt}l!{\vrule width 1.5pt}}{#1} % vertikaler Strich (1.5) LINKS & RECHTS der Zelle
}
\newcommand\Thickvrulel[1]{
  \multicolumn{1}{!{\vrule width 1.5pt}l}{#1}                      % vertikaler Strich (1.5) LINKS der Zelle
}
\newcommand\Thickvruler[1]{
  \multicolumn{1}{c!{\vrule width 1.5pt}}{#1}                      % vertikaler Strich (1.5) RECHTS der Zelle
}
\newcommand\Thinvrulel[1]{
  \multicolumn{1}{!{\vrule width 0.75pt}l}{#1}                     % vertikaler Strich (0.75) LINKS der Zelle
}
\newcommand\Thinvruler[1]{
  \multicolumn{1}{c!{\vrule width 0.75pt}}{#1}                     % vertikaler Strich (0.75) RECHTS der Zelle
}

\newcommand\stdrulel[1]{
	\multicolumn{1}{!{\vrule}c}{#1}
}
\newcommand\stdruler[1]{
	\multicolumn{1}{c!{\vrule}}{#1}
}
\newcommand\stdrulelinksb[1]{
	\multicolumn{1}{!{\vrule}l!{\vrule}}{#1}
}
\newcommand\stdrulecent[1]{
	\multicolumn{1}{!{\vrule}c!{\vrule}}{#1}
}
\newcommand\stdrulellinks[1]{
	\multicolumn{1}{!{\vrule}l}{#1}
}
\newcommand\stdrulerlinks[1]{
	\multicolumn{1}{l!{\vrule}}{#1}
}

\newcommand\RotText[1]{\rotatebox[origin=c]{90}{\parbox{1cm}{\centering#1}}}


\newcommand{\myitem}{\item\quad}                                  % horizontaler Abstand itemize
\newcommand{\ccol}{\cellcolor{black!20}}                          % CellColor Grau

\def\RA{\rlap{\scalebox{1.6}{$\rightarrow$}}}                     % Pfeile in Tabelle zeichnen stypackage/stackengine
\def\DA{\smash{\bclap{\scalebox{1.6}{$\downarrow$}}}}

\makeatletter                    
\def\hlinewd#1{
\noalign{\ifnum0=`}\fi\hrule \@height #1                         % Dicke der horizontalen Linie
\futurelet\reserved@a\@xhline}
\makeatother

\newcolumntype{Y}{>{\centering\arraybackslash}X}                 % ColumnType


% Entwurfslayout
\newcolumntype{V}[1]{%
  >{\begin{varwidth}[t]{\dimexpr\textwidth-\wd#1-2\fboxsep-2\fboxrule\relax}\arraybackslash}
    l<{\strut\end{varwidth}}}

\newsavebox\Breitestes % Box für längstes „item“
%Syntax: \Rahmen[<trenner>]{<längstes „item“>}{<tabelleninhalt>}
\newcommand\Rahmen[3][:\qquad]{%
  \sbox\Breitestes{#2#1}%
  \begin{flushleft}%
   {\renewcommand{\arraystretch}{1.2}\begin{tabular}{@{}l<{#1}@{}V{\Breitestes}@{}}
      #3
    \end{tabular}}%
  \end{flushleft}
}

%----------------------------------------------------------------------------------------------------------------------
% Abbildungen
%----------------------------------------------------------------------------------------------------------------------
\usepackage{graphicx}                                              % Einbinden von Grafiken mit Pfadangabe
\usepackage{float}                                                 % zum genauen platzieren von Grafiken und Tabellen
\graphicspath{{images/}}
\usepackage{pdfpages}                                              % Einbinden PDF-Dateien


%----------------------------------------------------------------------------------------------------------------------
% Inhaltsverzeichnis
%----------------------------------------------------------------------------------------------------------------------
%\usepackage[nottoc]{tocbibind}
\setcounter{secnumdepth}{3} 
\renewcommand*{\theparagraph}{\thesubsubsection.\alph{paragraph}} 
%\setcounter{secnumdepth}{3}
\setcounter{tocdepth}{3}
% ------ section fett im TOC
\usepackage{tocstyle}
\usetocstyle{allwithdot}
%\usetocstyle{KOMAlike}
\settocfeature[toc][1]{entryhook}{\protect\hspace*{-0.5em}\nobreakspace} % Section Horizontal in InhaltsVZ anpassen an Chapter
\settocfeature[toc][2]{entryhook}{\protect\hspace*{-0.9em}\nobreakspace} % Subsection Horizontal in InhaltsVZ anpassen an Section
\settocfeature[toc][3]{entryhook}{\protect\hspace*{-0.9em}\nobreakspace} % Subsection Horizontal in InhaltsVZ anpassen an Section
%\settocfeature[lof][1]{entryhook}{\protect\hspace*{-1.9em}\nobreakspace} % Kein horizontaler Einzug
%\settocfeature[lof][2]{entryhook}{\protect\hspace*{-1.9em}\nobreakspace} % Kein horizontaler Einzug
%\settocfeature[lot][1]{entryhook}{\protect\hspace*{-1.9em}\nobreakspace}
%\settocfeature[lot][2]{entryhook}{\protect\hspace*{-1.9em}\nobreakspace}


\usepackage{setspace}                                            % 1,5-zeiligen Zeilenabstand
\onehalfspacing 
%\BeforeStartingTOC[toc]{\singlespacing} 

%TIKZ zum Zeichnen
\usepackage{pgfplots}
\usepackage{tikz}
\usetikzlibrary{arrows,calc,positioning}
\usepackage[format=hang]{subfig} %margin=10pt,
%\usepackage{subcaption}
\pgfplotsset{compat=newest}
\usepgfplotslibrary{dateplot}

%----------------------------------------------------------------------------------------------------------------------
% Mathematische Formeln & Algorithmen
%----------------------------------------------------------------------------------------------------------------------
\usepackage[german,linesnumbered,vlined]{algorithm2e}
\usepackage{amsmath}
\usepackage{amssymb}
\usepackage{amstext}
\usepackage{amsfonts}
\usepackage{mathrsfs}
\usepackage{array}
\usepackage{mathtools}
\usepackage{wasysym}
\usepackage[a]{esvect}
\usepackage{units}
\usepackage{listings}
\lstset{breaklines=true,label=}
% \lstset{basicstyle=\ttfamily}
\newcommand{\code}[1]{\texttt{\hyphenchar%
\font45%
\sloppy%
\fontdimen2\font=0.4em%
\fontdimen3\font=0.2em%
\fontdimen4\font=0.1em%
\fontdimen7\font=0.1em%
#1}}

\SetAlgorithmName{Alg.}{Alg.}{Algorithmenübersicht}               % Algorithmusname


%----------------------------------------------------------------------------------------------------------------------
% Symbolverzeichnis
%----------------------------------------------------------------------------------------------------------------------
\newcommand{\Go}{\grqq\xspace} 	                                  % Gänsefüßchen oben
\usepackage[
toc,				                                              % fügt das Verzeichnis dem Inhaltsverzeichnis zu
acronym,			                                              % Abkürzungsverzeichnis hinzufügen
%numberline,		                                              % richtet Sortierung an Chapter aus
%description,		                                              % Beschreibung
nonumberlist,		                                              % keine Seitenzahlen
nopostdot]		                                                  % kein Punkt am Ende der Beschreibung
{glossaries}

\newglossary[slg]{symbols}{sls}{slo}{Symbolverzeichnis}

%Define custom style for glossaries
\newglossarystyle{mysymbstyle}{%
\renewcommand{\glossarypreamble}{\emph{Hinweis}: Bei der Angabe der Symbole soll sich auf die Wesentlichen beschränkt werden. Die jeweils zutreffende Bedeutung ergibt sich entweder aus dem Kontext oder ist explizit im Text angegeben.}
\renewenvironment{theglossary}% 
{
\begin{longtable}{@{}p{0.2\textwidth}@{}p{0.8\textwidth}}} 
{\end{longtable}}% 
\renewcommand*{\glossaryheader}{}% 
\renewcommand*{\glsgroupheading}[1]{}% 
\renewcommand*{\glossaryentryfield}[5]{% 
\glstarget{\textbf{##1}}{\sffett{\rlap{##2}}} & ##3\glspostdescription\space ##5\\} 
\renewcommand*{\glossarysubentryfield}[6]{% 
\glossaryentryfield{##2}{##3}{##4}{##5}{##6}}% 
\renewcommand*{\glsgroupskip}{ & \\} % 
\renewcommand{\arraystretch}{1.4} 

}


 %Define custom style for glossaries
\newglossarystyle{myacrstyle}{%
  % full stop after every description
    \renewcommand*{\glspostdescription}{}%
    % put the glossary in a longtable environment:
  % left alignment, no white in front und three columns
    \renewenvironment{theglossary}{\begin{longtable}[l]{@{}p{0.2\textwidth}@{}p{0.8\textwidth}}}{\end{longtable}}
    % have nothing after \begin{theglossary}:
    \renewcommand*{\glossaryheader}{}%
  % uncomment following line if you want headings
    % \renewcommand*{\glossaryheader}{\bfseries Symbol & \bfseries Unit & \bfseries Description \endhead}%
    % have nothing between glossary groups (next two commands):
    \renewcommand*{\glsgroupheading}[1]{}%
    % Suppress the vertical gap at the start of each group
    \renewcommand*{\glsgroupskip}{}%
    % set how each entry should appear:
    \renewcommand*{\glossaryentryfield}[5]{%
        \glstarget{##1}{##2}    % Name
        %& ##4                	    % Symbol
        & ##3                           % Description
        % & ##5                       % Page list
        \\% end of row
    }%
    % Sub entries treated the same as level 0 entries:
    \renewcommand*{\glossarysubentryfield}[6]{%
        \glossaryentryfield{##2}{##3}{##4}{##5}{##6}
    }%
}


\makeglossaries

\newglossaryentry{romanletter}{name={Lateinische Buchstaben}, description={}} %trennt Symbolverzeichnis in lateinische und griechische Buchstaben
\newglossaryentry{greekletter}{name={Griechische Buchstaben}, symbol={}, description={}} %trennt Symbolverzeichnis in lateinische und griechische Buchstaben
\newglossaryentry{mathletter}{name={Mathematische Operatoren},description={}} %trennt Symbolverzeichnis in lateinische und griechische Buchstaben
\newglossaryentry{model}{name={Formelzeichen},description={}}

%----------------------------------------------------------------------------------------------------------------------
% Sonstiges
%----------------------------------------------------------------------------------------------------------------------


% Änderung des Aufzählungszeichens
%\renewcommand{\labelitemi}{${\color{gray!80}\raisebox{.1ex}{\scalebox{.70}{$\blacksquare$}}}$}
\renewcommand{\labelitemi}{--}
%\renewcommand{\labelenumi}{\roman{enumi})} 


\lstset{
%xleftmargin= 15pt,
numbers=left,
numberstyle=\tiny,
language=java,
basicstyle=\footnotesize\sffamily,
frame=tb,
keepspaces=false,
breaklines=true,
columns=flexible,
showstringspaces=false,
captionpos=b,
commentstyle=\color{gray},
%float=[htb], 
literate=%
    {Ö}{{\"O}}1
    {Ä}{{\"A}}1
    {Ü}{{\"U}}1
    {ß}{{\ss}}1
    {ü}{{\"u}}1
    {ä}{{\"a}}1
    {ö}{{\"o}}1
    {~}{{\textasciitilde}}1
}

\makeatletter
\newlength{\linenumwidth} \setlength{\linenumwidth}{3em}% Redefine as required
\newlength{\numwidth}%
\setlength{\numwidth}{\widthof{\normalfont{\lst@numberstyle{999}}}}% Up to 2-digit (99) line numbers
\def\lst@PlaceNumber{%
  \makebox[\numwidth+1em][l]{%
    \makebox[\numwidth][r]{\normalfont\lst@numberstyle{\thelstnumber}}%
  }%
}
\makeatother

% Save the original way of printing the number (Custom Line Numbering)
\let\othelstnumber=\thelstnumber
\def\createlinenumber#1#2{
    \edef\thelstnumber{%
        \unexpanded{%
            \ifnum#1=\value{lstnumber}\relax
              #2%
            \else}%
        \expandafter\unexpanded\expandafter{\thelstnumber\othelstnumber\fi}%
    }
    \ifx\othelstnumber=\relax\else
      \let\othelstnumber\relax
    \fi
}



% Installation der Komponenten im Anhang (CommandLine-Syntax)
\newcommand{\cmdline}[1]{\texttt{> #1}}

\newcolumntype{C}[1]{>{\centering\arraybackslash}m{#1}}

% Fußnoten eingerückt untereinander
\deffootnote[]{1em}{1em}{\textsuperscript{\thefootnotemark\ }}

% fügt automatisch = hinzu bei formeln
\newenvironment{conditions}
  {\par\vspace{\abovedisplayskip}\noindent\begin{tabular}{>{$}l<{$} @{${}={}$} l}}
  {\end{tabular}\par\vspace{\belowdisplayskip}} 

% Fußnoten über alle Kapitel absolut
\usepackage{remreset}
\makeatletter
\@removefromreset{footnote}{chapter}
\makeatother


\setkomafont{dictumtext}{\itshape\footnotesize}                     % Fußnotengröße, kursiv
\renewcommand*{\dictumwidth}{.58\textwidth}                         % auf 65% der Textbreite
\renewcommand*{\raggeddictumtext}{\raggedleft}                      % Text nach links aufgeflattert → rechtsbündig
\renewcommand*{\dictumauthorformat}[1]{#1\vspace{12mm}}
%\setkomafont{dictumauthor}{\scshape}
\renewcommand*{\dictumrule}{}

%Definition hochgestelltes Copyright-Symbol
\def\CopTop{\textsuperscript{\textcopyright}}
\newcommand{\sffett}[1]{\textsf{\textbf{#1}}}
\newcommand{\mathsfbold}[1]{\boldsymbol{\mathsf{#1}}}


%Definition neuer Column-Type
\newcolumntype{v}[1]{%
>{\raggedright\hspace{0pt}}p{#1}%
}
\newcolumntype{R}{>{\raggedleft\arraybackslash}X}

% Zitat im Fließtext
\newcommand*{\zitat}[2]{% 
   \normalfont
   \begin{quote} #1 #2 
   \end{quote} 
   \normalsize 
} 

% Spezielles eingekringeltes Plus
\newcommand{\plus}{\text{{\large\textcircled{\normalsize \texttt{+}}}}}

% Farbe definieren
\definecolor{rotPOC}{RGB}{231 76 60}
\definecolor{blauPOC}{RGB}{0 0 255}

% Abkürzungsverzeichnis Definition der Schlüsselwörter
\let\acrlongdat\glsuseri
\let\acrlongplengl\glsuserii
\let\acrlongengl\glsuseriii

% pifont-Definitionen
\newcommand{\xmark}{\ding{55}}%
\newcommand{\quadrat}{\ding{110}}%
\newcommand{\kreis}{\ding{108}}%


%----------------------------------------------------------------------------------------------------------------------
% Silbentrennung unbekannter Wörter
%----------------------------------------------------------------------------------------------------------------------
%\hyphenation{ger-ing-er}
%\hyphenation{Ein-zel-kom-po-nen-ten}
%\hyphenation{Stör-an-fäl-lig-keit}
%\hyphenation{ein-fach-e}

%----------------------------------------------------------------------------------------------------------------------
% Listingseinstellungen und Modelica Syntax Highlighting
%----------------------------------------------------------------------------------------------------------------------
\lstdefinelanguage{Modelica}
{
morekeywords=[1]{
algorithm,and,annotation,as,assert,block,break,case,class,connect,connector,
constant,constrainedby,der,discrete,each,else,elseif,elsewhen,encapsulated,
end,enumeration,equality,equation,expandable,extends,external,failure,final,
flow,for,function,guard,if,import,in,initial,inner,input,List,local,loop,
match,matchcontinue,model,not,operator,Option,or,outer,output,package,parameter,
partial,protected,public,record,redeclare,replaceable,return,stream,
subtypeof,then,Tuple,type,uniontype,when,while},
morekeywords=[2]{true, false},
morekeywords=[3]{Real, Integer,Modelica.Blocks.Interfaces.RealInput,Modelica.Blocks.Interfaces.IntegerInput},
% Do not make true,false keywords because fn(true,x, false ) shows up as fn(true,x, *false*)
sensitive=true,
comment=[l][\color{stringcolor}]{//},
morecomment=[s]{/*}{*/},
alsodigit={.,-},
morestring=[b]',
morestring=[b]",
morestring=[s]{/*}{*/},
}
[keywords,comments,strings]
\definecolor{keywordcolor1}{rgb}{0,0,.4}
\definecolor{keywordcolor2}{rgb}{.90,0,0}
\definecolor{keywordcolor3}{rgb}{.90,0,0}
\definecolor{stringcolor}{rgb}{0.133,0.545,0.133}
% \definecolor{listingbgcolor}{rgb}{0.95,0.95,0.95}

\lstset{
breaklines=true,
language=Modelica,
basicstyle=\scriptsize\ttfamily,
keywordstyle=[1]\color{keywordcolor1}\bfseries,
keywordstyle=[2]\color{keywordcolor2},
keywordstyle=[3]\color{keywordcolor2},
stringstyle=\color{stringcolor},
numbers=left,                   % where to put the line-numbers
numberstyle=\footnotesize\color{gray},  % the style that is used for the line-numbers
stepnumber=2,                   % the step between two line-numbers. If it's 1, each line 
                                  % will be numbered
numbersep=5pt,                  % how far the line-numbers are from the code
backgroundcolor=\color{white},
framexleftmargin=5pt,
xleftmargin=5pt,
xrightmargin=5pt,
showstringspaces=false,
showspaces=false,               % show spaces adding particular underscores
showtabs=false,                 % show tabs within strings adding particular underscores
frame=single,                   % adds a frame around the code
rulecolor=\color{black},        % if not set, the frame-color may be changed on line-breaks within not-black text (e.g. comments (green here))
tabsize=2,                      % sets default tabsize to 2 spaces
captionpos=b,                   % sets the caption-position to bottom
breaklines=true,                % sets automatic line breaking
breakatwhitespace=false,        % sets if automatic breaks should only happen at whitespace
aboveskip = \floatsep,
}
