%
% Schlussbetrachtung
%
% @version 1.0
% @author wipatrick
% @created 22. November 2015
% @edited 

\setchapterpreamble[o]{%
\dictum[--- \textsc{René Descartes}, \emph{französischer Philosoph, Mathematiker und Naturwissenschaftler}]{\Gun Alles Wissen besteht in einer sicheren und klaren Erkenntnis. \Gob}}
\renewcommand{\chapterheadstartvskip}{\vspace*{3cm}}

\chapter{Schlussbetrachtung}
\label{chap:schlussteil}
\renewcommand{\chapterheadstartvskip}{\vspace*{-0.5cm}}

\section{Fazit}
\label{sec:zusammenfassung}



\section{Ausblick und Ansatzpunkte für weitere Arbeiten}
\label{sec:ausblick}

Dazu werden im ersten Schritt werden auf Basis von aktuellen Messungen des Systemzustandes und der vorgegebenen Steuergrößen zu jedem diskreten Steuerungszeitpunkt optimale Steuerungspläne bestimmt werden..
Im nächsten Schritt soll die Modellprädiktive Regelung Vorhersagewerte für den äußeren, nicht beeinflussbaren Steuergrößen zur die Bestimmung der optimalen Steuersignale nutzen.

Welchge art der Verwendung?
MPC mit JModelica.org also deren mpc klasse
eigene in casadi
etc?

Vergleich mit bestehenden Modellen möglich z.B. das THERAKLES Modell

Annahme Temperatur und homogen im Raum unteruschen

Sonnenstarhlung genauer untersuchen und für MPc auf Vorhersagewerte gehen.
Implementierung MPC, länge intervall, kostenfunktionen versch. kriterien untersuchen