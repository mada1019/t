%
% Schlussbetrachtung
%
% @version 1.0
% @author wipatrick
% @created 22. November 2015
% @edited 

\setchapterpreamble[o]{%
\dictum[--- \textsc{René Descartes}, \emph{französischer Philosoph, Mathematiker und Naturwissenschaftler}]{\Gun Alles Wissen besteht in einer sicheren und klaren Erkenntnis. \Gob}}
\renewcommand{\chapterheadstartvskip}{\vspace*{3cm}}

\chapter{Schlussbetrachtung}
\label{chap:schlussteil}
\renewcommand{\chapterheadstartvskip}{\vspace*{-0.5cm}}

\section{Fazit}
\label{sec:zusammenfassung}

Das Ziel dieser Arbeit war es, herauszufinden wie eine Anlage zur Raumtemperaturregelung und ein mathematisches Modell derselben aufgebaut sein können, um eine \acrlong{mpr} der Raumtemperatur zu ermöglichen.
Zur Beantwortung dieser Frage wurde im Verlauf der Arbeit eine konkrete Anlage installiert und ein Raummodell entwickelt.

Im dritten Kapitel wurden die besonderen Anforderungen für eine Modellprädiktive Regelung analysiert und daraus eine erste Idee abgeleitet.Diese bestand in der Automatisierung eines Raumes der Hochschule Karlsruhe, der zukünftig als Forschungsumgebung genutzt werden kann. Durch die Anbringung von einfachen Temperatursensoren und den Einsatz eines Stellantriebs zur Ansteuerung des bestehenden Heizkörpers innerhalb des Raumes, wurde die Idee in die Praxis umgesetzt. Die Kommunikation innerhalb der Anlage wurde durch eine klare Struktur möglichst einfach und übersichtlich gehalten. Durch die Verwendung von Interfaces wird auch ohne detaillierte Fachkenntnisse über den Aufbau der Anlage und den verwendeten Kommunikationstechnologien eine einfache Ansteuerung derselben ermöglicht. Abschließend wurde eine Zweipunktregelung zur Inbetriebnahme der Anlage eingesetzt, anhand derer die Fähigkeit zur Regelung einer Raumtemperatur aufgezeigt wurde.

Damit war es möglich die aufgestellte Hypothese zu belegen: Die Anlage eignet sich nachweislich zur Regelung einer Raumtemperatur.

Der anschließende Teil beschäftigte sich mit der Bildung eines Raummodells. Das Modell wurde für den Betrieb einer Modellprädiktiven Regelung der zuvor installierten Anlage entwickelt. Die daraus resultierenden Anforderungen wurden bei dem schrittweise erfolgten Aufbau des Raummodells beachtet. Bei einer Validitätsprüfung wurde festgestellt wurde, dass die grundlegende Dynamik der realen Prozesse ausreichend beschrieben wird. Anschließend wurde eine Parameterschätzung mithilfe von \cite{casiopeia} durchgeführt, um die Modellgüte zu verbessern. Nachdem das Modellverhalten anhand von weiteren Simulationen untersucht wurde, konnte seine prinzipielle Eignung für den Einsatz mit einer Modellprädiktiven Regelung bestätigt werden. Abschließend wurde durch eine Übersetzung des Modells in ein optimierungsfähiges Objekt gezeigt, dass es sich für eine Modellprädiktive Regelung in der Plattform \textsc{JModelica.org} eignet.


Dadurch konnte eine weitere Hypothese belegt werden: Das Modell eignet sich für eine Modellprädiktive Regelung in der Plattform \textit{JModelica.org}.

Anhand der in dieser Arbeit vorgestellten Ergebnisse und durch die Bestätigung der beiden Thesen, kann die Forschungsfrage durch die installierte Anlage und das gebildete Modell beantwortet werden.

Zunächst wurde die Grundvoraussetzung geschaffen, indem eine Anlage zur Regelung der Raumtemperatur installiert und deren Fähigkeit zur Regelung bestätigt wurde. Diese wurde zusätzlich mit einer zentralen Bedienoberfläche in Python ausgestattet, die durch ihren Aufbau und eine klare Struktur eine einfache Möglichkeit der Ansteuerung bietet. Dadurch eignet sich die Anlage in besonderem Maße für eine Modellprädiktive Regelung mit \textit{JModelica.org} und kann daher auch für eine Modellprädiktive Temperaturregelung genutzt werden. 
Durch verschiedene Simulationen und Schätzungen wurde gezeigt, dass das Modell eine hinreichende Beschreibung des realen Systemverhaltens liefert. Gleichzeitig wurde die Komplexität des Modells bei der Entwicklung möglichst gering gehalten, sodass eine moderate Rechenkapazität zur Simulation des Modells ausreicht. Abschließend konnte das Modell fehlerfrei in zwei optimierungsfähige Objekte der Plattform \textit{JModelica.org} übersetzt werden. Damit wurde die Kompatibilität des Modells für einen Einsatz mit der Modellprädiktiven Regelung gezeigt.

Eine Anlage mitsamt einem ihr zugehörigen Modell für eine Modellprädiktive Temperaturregelung kann also aus der in dieser Arbeit beschriebenen Anlage und dem aufgestellten Modell aufgebaut sein.

\section{Ausblick und Ansatzpunkte für weitere Arbeiten}
\label{sec:ausblick}

Nachdem die Fähigkeit zur Modellprädiktiven Regelung der Anlage und des Modells gezeigt wurde, bietet sich als nächster Schritt eine tatsächliche Implementierung einer Modellprädiktiven Temperaturregelung an. Dafür gilt es zu klären, ob hierzu die bestehende, generische Klasse von \textit{JModelica.org} genutzt werden kann oder ob es sinnvoll ist eine eigene Implementierung der Modellprädiktiven Regelung durch Nutzung von \textit{CasADi} Werkzeugen durchzuführen. Ein wichtiger Aspekt dabei ist die Übergabe der gemessenen Steuergrößen an die Optimierungsumgebung, welche durch die Umwelt vorgegeben sind. Darüber hinaus stellt sich die Frage, ob der Einbezug von Prognosewerten für Steuergrößen, wie beispielsweise der Globalstrahlung oder der Außenlufttemperatur, sinnvoll sind. 

Ein weiteren Ansatzpunkt für zukünftige Arbeiten bietet das Raummodell, bei welchem durch einen Vergleich mit bestehenden umfangreicheren Modellen, wie beispielsweise \cite{therakles13}, Verbesserungspotenziale aufgedeckt werden können.

Es ist ebenso denkbar, eine detaillierte Untersuchung der Raumtemperaturverteilung und des exakten Einflusses der Solarstrahlung auf die Raumtemperatur vorzunehmen, unter anderem um die angenommene Homogenität der Temperatur innerhalb eines Raumes zu überprüfen.

Die Anlage bietet aufgrund ihres modularen Aufbaus die Möglichkeit für Erweiterungen, beispielsweise durch eine Klimatisierung und eine automatisierte Ansteuerung einer Verschattung. Dadurch wäre sie in der Lage eine ganzjährige Regelung der Raumtemperatur zu bewirken. 
