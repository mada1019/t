%
% Anlagendesign
%
% @version 1.0
% @author dmayer
% @created 29. Dezember 2015

\chapter{Modellbildung des Raumes}
\label{chap:modellbildung}

\renewcommand{\chapterheadstartvskip}{\vspace*{-0.5cm}}

\section{Das einfache Raummodell}

Das Modell soll zunächst so simpel wie möglich gestaltet werden um eine Optimierung mit Hilfe von MPC zu ermöglichen. Dessen Verfahren zur Optimierung sind gradientenbasiert und erfordern damit die Erzeugung von stetigen Ableitungen bis zum zweiten Grad. Daher soll die Komlpexität des Modells zunöchgst sehr gering gehalten werdeen und dann Stück für Stück erhöht werden und die damit die Geanuiogkeit des modells erhöht werden

Systemgrenzen/modell


Die Abgrenzung bzw Wahl der Grenzen zur Bilanzierung eines thermodynamischen Systems erfolgt nach dem gesuchten Zustand und der gesuchten Zustandsgröße, der Raumtemperatur im Raum K004b. Der gesucten hießt zu berechnenenden/unbekannten

Dazu lässt sich das thermodynamische System in drei Teile gliedern. Zum einen in den zu untersuchenden Raum, der durch die Aussenwände des Raumes begrenzt wird und innerhalb dessen Grenzen die zu betsimmende Raumtemperatur vorherrscht. Zum anderen die Teilsysteme Gebäude und die Umgebung, innerhalb deren Grenzen jeweils auch eine chrakterisierende Temperatur vorherrscht. Die Systemgrenzen des Raumes werden als geschlossen angesehen, dass heißt das Öffnen und schließen von Fenstern und Türen wird nicht explizit berücksichtigt sondern kann nur nur implizit als Störgröße berücksichtigt werden. Die Grenzen zwischen den Teilsystemen 

Im Raumomodellfall ändert sich also immer der GG-Zustand bezogen auf die Temperatur vom Raum und wird durch die wechselnde Temperatur der äußeren Umgebung bedingt.
ref erfahrungssatz
t 
\newpage

\section{Erweiterung durch Sonneneinstrahlung}

\newpage
t 
\newpage

\section{Erweiterung durch Heizkörper}
t 
\newpage
\section{Validierung des Modells}

\section{Anpassung des Modells mit Parameterschätzung}