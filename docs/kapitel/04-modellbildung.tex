%
% Modellbildung
%
% @version 1.0
% @author dmayer
% @created 29. Dezember 2015

\setchapterpreamble[o]{%
\dictum[--- \textsc{Norbert Wiener}]{\Gun Das beste Modell für eine Katze ist eine Katze; möglichst dieselbe Katze. \Gob}}
\renewcommand{\chapterheadstartvskip}{\vspace*{2cm}}

\chapter{Modellbildung des Raumes}
\label{chap:modellbildung}

Ziel dieses Kapitels ist es, ein hinreichend exaktes Modell zur Berechnung der Raumtemperatur, basierend auf den thermodynamischen Prozessen mit dessen Umgebung und der Anlage aus Kapitel \ref{chap:anlagendesign}, zu bilden, um damit und unter Zuhilfenahme der Anlage Modellpräditive Regelung zu ermöglichen.
Dazu wird zunächst ein einfaches Grundmodell für einen hypothetischen Raum gebildet, dass anschließend schrittweise an den bestehenden Raum erweitert angepasst wird, bis eine die Qualität des Modells ausreichend ist.

Physikalisch motiviert.
Anhand von Zustand, Steuergröße und Parameter/Eigenschaft

\section{Anforderungen an das Raummodell}
Da MPC die Lösung von Nichtlinearen Gleichungssystemen erfordert wird ein erhöhter Rechenbedarf benötigt. Daher sollte das Modell so einfach wie möglich gehalten werden. Die Krux liegt also darin, einen geeigneten Kompromiss zwischen Komplexität und Genauigkeit des Modells zu finden, der eine sinnvolle MPC Regelung ermöglicht.
Des Weiteren werden gradientenbasierte Ableitungen bei der Optmierung/Lösung des LGS generiert weshalb auch keine Unstetigkeiten im Modell vorkommen dürfen.
Die triviale Aufgabe ist die hinreichend genaue Beschreibung der Realität bzw realen Vorgänge. hinreichend definieren.
Damit eine Steuerung mit Hilfe von Modellprädiktiver Regelung möglich ist, darf das Modell keine hohe Kompliziertheit aufweisen und sollte durch möglichst wenig Gleichunge trotzdem ein möglichst genaues Abbild der Realität abbilden.
Das Modell soll zunächst so simpel wie möglich gestaltet werden um eine Optimierung mit Hilfe von MPC zu ermöglichen. Dessen Verfahren zur Optimierung sind gradientenbasiert und erfordern damit die Erzeugung von stetigen Ableitungen bis zum zweiten Grad. Daher soll die Komlpexität des Modells zunöchgst sehr gering gehalten werdeen und dann Stück für Stück erhöht werden und die damit die Geanuiogkeit des modells erhöht werden

\section{Das Grundmodell des Raumes}

Um ein möglichst einfaches Grundmodell zu erhalten, wird zunächst ein hypothetischer Raum betrachtet. Dieser Raum bildet zusammen mit der ihn umgebenden Luft ein abgeschlossenes thermodynamisches System, wie in Kapitel \ref{sec:grundlagenmodell} beschrieben. Der Raum ist selbst mit Luft gefüllt und wird zu allen sechs Seiten hin durch Wände begrenzt. Damit bildet der Raum ein geschlossenes System, da keine Massenströme über die Grenzen hinweg fließen können. An den Grenzflächen kann also lediglich Wärme zwischen der Umgebung und dem  Raum ausgetauscht werden. Des Weiteren wird eine homogene Temperatur innerhalb des Raumes und der Umgebung angenommen, welche in der Realität eingeschwungenen Gleichgewichtszuständen innerhalb der beiden Teilsysteme entspricht. Um die Annahme für den Raum zu überprüfen, muss noch festgestellt werden auf welcher zeitlichen Skala der Einschwingvorgang für eine homogene Temperatur innerhalb des Raumes stattfindet und ob dieser damit eine Relevanz für die Modellbildung besitzt.

\begin{figure}
\centering
\includegraphics[width=\textwidth]{abbildungen/20160316_grundraum}
\caption{Grundmodell eines Raumes}
\label{fig:grundraum}
\end{figure}

Zur Bestimmung der Temperatur innerhalb des Raumes, ausgehend von einer initialen Raumtemperatur und dem externen Steuerungsparameter der Umgebungstemperatur, muss der Ausgleichsprozess zwischen Raum und Umgebung untersucht werden, konkret der ausgetauschte Wärmestrom. Um diesen nach \ref{eq:qdot} zu berechnen, müssen zunächst die verschiedene modellrelevanten Eigenschaften des Raumes durch physikalische Größen und Variablen beschrieben werden. Zur Berechnung der Austauschoberfläche wird die Raumbreite, -länge und -höhe benötigt und weiterhin sind der U-Wert einer Betonwand, die spezifische Wärmekapazität und Dichte von Luft für die Bestimmung des Wärmestroms relevant.
Diese modellrelevanten Eigenschaften sind allesamt mit ihren Zahlenwerten in Tabelle \ref{tab:eigenschaften_raum} zusammengefasst.

\begin{table}[H]
\centering
\small
\renewcommand{\arraystretch}{1.3}
\begin{threeparttable}
\begin{tabularx}{1\textwidth}{p{0.5\textwidth}m{0.2\textwidth}m{0.18\textwidth}}
\toprule
\textbf{Modellrelevante Eigenschaften} & \textbf{Wert} & \textbf{Einheit} \\
\cmidrule[0.5pt](r{0.25em}){1-1} 
\cmidrule[0.5pt](l{0.25em}){2-2}
\cmidrule[0.5pt](l{0.25em}){3-3}

Raumbreite & 7,81\tnote{1)} & $[m]$ \\ 
\ccol Raumlänge & \ccol 5,78\tnote{1)} & \ccol $[m]$ \\
Raumhöhe & 2,99\tnote{1)} & $[m]$ \\
\ccol Wärmedurchgangskoeffizient Betonwand & \ccol 2,0\tnote{2)} & \ccol $[\frac{W}{m^{2}*K}]$\\
Spezifische Wärmekapazität von Luft & 1.000,0\tnote{3)} & $[\frac{J}{kg*K}]$\\
\ccol Dichte von Luft & \ccol 1,25 \tnote{3)} & \ccol $[\frac{kg}{m^{3}}]$\\
\bottomrule
\end{tabularx}
\begin{tablenotes}[]\footnotesize\singlespacing\setlength\labelsep{0pt}
\item[1)] Werte durch eigene Vermessung des Raumes K004b vom 07.12.2015.
\item[2)] Schätzwert, geschätzt nach \cite[S.~409]{re14} mit Richtwerten aus \cite[S.~194ff.]{re14}.
\item[3)] Tabellenwert aus \cite[S.~139]{ha13}.
\end{tablenotes}
\end{threeparttable}
\caption{Eigenschaften des Raummodells}
\label{tab:eigenschaften_raum}
\end{table}

Erfolgt nun die Bilanzierung des Raumes mit Hilfe des ersten Hauptsatzes der Thermodynamik nach \ref{eq:hauptsatz} und die Berechnung der inneren Energie des Raumes nach \ref{eq:innereenergie} ergibt sich folgendes, einfaches Gleichungssystem zur Bestimmung der Raumtemperatur in Abhängigkeit vom Steuergrößen Außentemperatur im Grundmodell in Modelica:

\begin{lstlisting}[language=Modelica,caption={Einfaches Gleichungssystem für das Grundmodell des Raumes in Modelica}, label=lst:grundraum]
equation
   /* calculate room volume */
   room_volume = room_length * room_height * room_breadth;
   /* calculate room mass */
   room_mass = room_volume * rho_air;
   /* calculate surface of heat exchange */
   exchange_surface = 2 * (room_length * room_breadth) + 2 * (room_length * room_height) + 2 * (room_breadth * room_height);
   /* calculate inner energy*/
   room_u = room_mass * cp_air * room_temperature;
   /* calculate derivative of the inner energy */
   der(room_u) = environment_qdot;
   /* calculate heatflow between room and environment */
   environment_qdot = u_wall * exchange_surface * (environment_temperature - room_temperature);
\end{lstlisting}

Damit ist ein Grundmodell für einen Raum gebildet, wie in \ref{fig:grundraum} graphisch dargestellt, um die Temperatur innerhalb eines Raumes zu berechnen. Dieses wird im Folgenden nun schrittweise erweitert und zum Abschluss überprüft, ob es der Realität genüge zu trägt.


\section{Modellerweiterung durch Berücksichtigung der realen Umgebung}

Im nächsten Schritt wird das einfache Raummodell zunächst an die reale Umgebung des Raums K004b angepasst. Die Lage von K00b ist in \ref{fig:skizzek004a} ersichtlich und es ist zu erkennen, dass der Raum lediglich zwei Außenwände besitzt, die an die Umgebungsluft grenzen: Die Wände in Richtung Süden und Westen. Die anderen beiden Wände, sowie die Decke und der Boden, grenzen an weitere Gebäudeteile des K-Gebäudes. Somit entspricht der Raum im Modell nach wie vor einem geschlossenen System und bildet weiterhin, zusammen mit dem umgebenden K-Gebäude und der Umgebungsluft, ein abgeschlossenes System. Jedoch müssen nun potenziell verschiedene Wärmeströme zwischen dem Raum und der Außenumgebung sowie dem Raum und dem K-Gebäude betrachtet werden. Da die fließenden Wärmeströme im Vergleich zur sehr großen Energie innerhalb des gesamten K-Gebäudes und der Umgebung nur verschwindend gering sind, wird der erwärmende beziehungsweise kühlende Effekt der Wärmeströme auf die beiden Teilsysteme vernachlässigt und es wird von konstanten, homogenen Temperaturen beider ausgegangen.

Durch diese Erweiterung des Modells hängt die Raumtemperatur nun von zwei Wärmeströmen und damit indirekt von zwei externen Steuergrößen, den Temperaturen in der Umgebung und im K-Gebäude, ab. Um die Wärmeströme separat berechnen zu können, wird die gesamte Oberfläche zum Wärmeaustausch aufgeteilt in die Austauschoberfläche mit der Umgebung und die Austauschoberfläche mit dem K-Gebäude. Des Weiteren werden im Modell die Temperatur der Außenumgebung und die Temperatur innerhalb des K-Gebäudes als externe Steuergrößen berücksichtigt. Das Gleichungssystem des Grundmodells in \ref{lst:grundraum} erweitert sich also um folgende Änderungen:

\begin{lstlisting}[language=Modelica, caption={Erweitertes Gleichungssystem Modell des Raumes unter Berücksichtigung der realen Umgebung in Modelica}, label=lst:raumeins]
equation
   [...]
   /* calculate surface of heat exchange with the environment */
   environment_surface = room_length * room_height + room_breadth * room_height;
   /* calculate surface of heat exchange with the remaining building */
   building_surface = 2 * (room_length * room_breadth) + room_length * room_height + room_breadth * room_height;
   /* calculate derivative of the inner energy */
   der(room_u) = environment_qdot + building_qdot;
   /* calculate heatflow between room and environment */
   environment_qdot = u_wall * environment_surface * (environment_temperature - room_temperature);
   /* calculate heatflow between room and building */
   building_qdot = u_wall * building_surface * (building_temperature - room_temperature);
\end{lstlisting}

Damit wurde das Raummodell an die reale Umgebung angepasst und um die Temperatur innerhalb des Raumes zu bestimmen, wird nun neben der Ausgangstemperatur im Raum und der Umgebungstemperatur noch die Temperatur innerhalb des restlichen K-Gebäudes berücksichtigt. Im nächsten Schritt werden die realen, räumlichen Gegebenheiten im Modell abgebildet.


\section{Modellerweiterung durch Berücksichtigung der räumlichen Gegebenheiten}

Um das Modell an die realen Gegebenheiten des Raumes K004b anzupassen, müssen zwei bauliche Gegebenheiten beachtet werden. Wie in \ref{fig:skizzek004a} dargestellt ist, ist in der südlichen Außenwand eine Fensterfront vorhanden. Da der U-Wert eines Fensters erheblich von dem U-Wert einer Wand abweicht, entsteht ein zusätzlicher Wärmestrom zwischen dem Raum und der Umgebung durch das Fenster hindurch. Das Öffnen und Schließen der Fenster mit daraus resultieren Massenströmen wird jedoch zunächst nicht explizit berücksichtigt, weshalb das Raummodell weiterhin als geschlossenes System betrachtet wird. Des Weiteren ist es möglich, den Raum über einen Heizkörper zu beheizen. Mit dem Heizkörper, der zunächst als einfache Wärmequelle im Modell ergänzt wird, erhöht sich die Anzahl der externen Steuergrößen erneut, da die Temperatur innerhalb des Raumes auch von dieser abhängig ist.

Durch diese Erweiterungen werden auch weitere physikalische Größen zur Beschreibung der Eigenschaften des Raummodells benötigt. Wie bereits erwähnt werden die Eigenschaften um den U-Wert eines Fensters, sowie die Breite und Höhe der Fensterfront ergänzt wie in Tabelle \ref{tab:eigenschaften_raumerw} zusammengefasst.

\begin{table}[H]
\centering
\small
\renewcommand{\arraystretch}{1.3}
\begin{threeparttable}
\begin{tabularx}{1\textwidth}{p{0.5\textwidth}m{0.2\textwidth}m{0.18\textwidth}}
\toprule
\textbf{Modellrelevante Eigenschaften} & \textbf{Wert} & \textbf{Einheit} \\
\cmidrule[0.5pt](r{0.25em}){1-1} 
\cmidrule[0.5pt](l{0.25em}){2-2}
\cmidrule[0.5pt](l{0.25em}){3-3}

Fensterbreite & 7,0\tnote{1)} & $[m]$ \\ 
\ccol Fensterhöhe & \ccol 2,08\tnote{1)} & \ccol $[m]$ \\
Wärmedurchgangskoeffizient Glas & 4,0\tnote{2)} & $[\frac{W}{m^{2}*K}]$\\

\bottomrule
\end{tabularx}
\begin{tablenotes}[]\footnotesize\singlespacing\setlength\labelsep{0pt}
\item[1)] Werte durch eigene Vermessung des Raumes K004b vom 07.12.2015.
\item[2)] Tabellenwert, geschätzt nach \cite[S.~270ff.]{h2000}.
\end{tablenotes}
\end{threeparttable}
\caption{Weitere Eigenschaften des Raummodells}
\label{tab:eigenschaften_raumerw}
\end{table}
 
Durch diese Anpassung verändert sich die Austauschoberfläche mit der Umgebung, die sich nun auf zwei Flächen mit verschiedenen Wärmedurchgangskoeffizienten verteilt. Des Weiteren wird eine Wärmequelle für die Heizung ergänzt, so dass sich folgende Änderungen des Gleichungssystems im Vergleich zum bisherigen Modell in \ref{lst:grundraum} ergeben:

\begin{lstlisting}[language=Modelica, caption={Erweitertes Gleichungssystem Modell des Raumes unter Berücksichtigung der räumlichen Gegebenheiten in Modelica},label=lst:raumzwei]
equation
   [...]
   /* calculate surface of heat exchange with the environment */
   environment_surface = room_length * room_height + room_breadth * room_height - window-surface;
   /* calculate surface of heat exchange with the remaining building */
   building_surface = 2 * (room_length * room_breadth) + room_length * room_height + room_breadth * room_height;
   /* calculate surface of window with the environment */
   window_surface=(window_length*window_height);
   /* calculate derivative of the inner energy */
   der(room_u) = environment_qdot + building_qdot + window_qdot + radiator_qdot;
   /* calculate heatflow between room and environment through the walls */
   environment_qdot = u_wall * environment_surface * (environment_temperature - room_temperature);
   /* calculate heatflow between room and environment through the window */
   building_qdot = u_glass * window_surface * (environment_temperature - room_temperature);
   /* calculate heatflow between room and building */
   building_qdot = u_wall * building_surface * (building_temperature - room_temperature);
\end{lstlisting}

Damit ist das Modell auch an die räumlichen Gegebenheiten angepasst und beschreibt dadurch in groben Zügen die realen Zusammenhänge. Allerdings kann eine Steuerung/Regelung eine Heizung nicht einfach beliebig als einfache Wärmequelle einsetzen, weshalb eine detailliertere, physikalisch motivierte Modellierung folgen wird. Die bisherigen Zusammenhänge des Modells sind in \ref{fig:raumeins} graphisch dargestellt. 

\begin{figure}
\centering
\includegraphics[width=\textwidth]{abbildungen/20160316_raumeins}
\caption{Erweitertes Raummodell}
\label{fig:raumeins}
\end{figure}

Bereits bei den Einsatzzielen der Anlage in \ref{sec:anforderungen} war gefordert, den Zusammenhang zwischen der Sonneneinstrahlung und der Raumtemperatur zu untersuchen sowie Störgrößen explizit in Kauf zu nehmen. Daher ist es passend, dass die Fensterfront in Richtung Süden ausgerichtet ist und der Raum K004b als Büro genutzt wird.
Daher wird im nächsten Abschnitt das Modell an diese beiden Faktoren angepasst.


\section{Modellerweiterung durch Berücksichtigung von Störgrößen und der Sonneneinstrahlung}

Der Raum K004b wird regulär als Büro genutzt, weshalb verschiedene Faktoren als Störgrößen in Bezug auf die Raumtemperatur betrachtet werden können. Zum einen wird durch die Menschen und deren Rechner weitere Wärme in den Raum eingebracht und zum Anderen werden die Fenster und die Türen manchmal geöffnet. Des Weiteren trifft insbesondere auf die südseitigen Fenster Sonnenstrahlung, welche ebenfalls Wärme in den Raum einbringen und damit einen Einfluss auf die Raumtemperatur haben.

\begin{lstlisting}[language=Modelica, caption={Erweitertes Gleichungssystem Modell des Raumes unter Berücksichtigung der Sonneneinstrahlung und Störgrößen},label=lst:raumdrei]
equation
   [...]
   /* calculate derivative of the inner energy */
   der(room_u) = environment_qdot + building_qdot + window_qdot + radiator_qdot + sun_qdot + otherfactors_qdot;
  
\end{lstlisting}

%Here We GO with model extension
\begin{figure}
\centering
\includegraphics[width=\textwidth]{abbildungen/20160317_raumzwei}
\caption{Erweitertes Raummodell}
\label{fig:raumdrei}
\end{figure}

\section{Validierung des Modells}

\section{Anpassung des Modells mit Parameterschätzung}