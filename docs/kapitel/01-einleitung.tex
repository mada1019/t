\setchapterpreamble[o]{%
\dictum[--- \textsc{Johann Wolfgang von Goethe}]{\Gun Erfolgreich zu sein setzt zwei Dinge voraus: Klare Ziele und den brennenden Wunsch, sie zu erreichen.\Gob}}
\renewcommand{\chapterheadstartvskip}{\vspace*{2cm}}

\chapter{Einleitung}
\label{chap:einleitung}
\pagenumbering{arabic}
\setcounter{page}{1}
\renewcommand{\chapterheadstartvskip}{\vspace*{-1cm}}


\section{Motivation und Problemstellung}
\label{sec:motivation}

Die neueren Entwicklungen in den südlichen und östlichen Staaten Asiens, allen voran China und Indien, haben einen starken Einfluss auf den enormen Anstieg des weltweiten, zukünftigen Energiebedarfs. In einem zentralen Prognoseszenario, schätzt die Internationale Energie-Agentur den Energieverbrauch im Jahr 2040 um ein Drittel höher ein als im vergangenen Jahr 2015.  (World Energy outlook 2015 S1)

Zudem sieht das Intergovermental Panel on climate change im Climate Change Report 2014 den Menschen als eine der Hauptursachen für den Klimawandel, insbesondere für die globale Erderwärmung (Climatechangereport 2014 S. V). Eine Schlüsselrolle bei der globalen Erwärmung nimmt der Ausstoss von Treibhausgasen ein. Besonders bei der Erzeugung von Elektrizitäts- und Wärmeenergie werden große Mengen an CO2 freigesetzt.
(Climatechangereport 2014 S. 47)

Daher wird der Bedeutung der Energieerzeugung auch in Zukunft eine immer weiter steigen/zunehmen. Den erneuerbaren Energien, welche ohne den Ausstoss schädlicher Treibhausgase auskommen, bietet sich damit ein enorme Chance. Dies wird beispielsweise durch die Zielsetzung der Bundesregierung in Deutschland belegt, deren Ziel es ist im Jahre 2050 den Energiebedarf zu 80 prozent aus Erneuerbaren Energien zu decken (Bilanz Energiewende Regierung S.2)
Als Erneuerbare Energien werden Energiequellen bezeichnet, die die nach menschlichem Zeithorizont unerschöpflich sind, worin sich ihr enormes Potenzial befindet. Sie umfassen die Planetenenergie durch Gravitation und die geothermische Energie, das mit Abstand größte Energieangebot bietet jedoch die Sonnenenergie.
Das Angebot an Sonnenenergie übersteigt den gesamten, weltweiten Energiebedarf um ein vielfaches und könnte diesen theoretisch vollständig decken, worin/womit sich ihr gewaltiges Potenzial begründet.

Um dieses Potenzial zu nutzen und den Weltenergiebedarf aus Regenerativen Energiequellen zu decken, gilt es jedoch noch einige Probleme zu lösen. Eines davon ist die Frage, wie die sich die technische Umwandlung der Solarenergie in eine nutzbare Energieform wie beispielsweise Wärme oder Elektrizität realisieren und weiter optimieren lässt. Wichtige Technologien hierzu sind die solarthermischen Kraftwerke, welche die Solarenergie zunächst in Wärme und anschließend in elektrische Energie umwandeln, sowie Solarkollektoren und -zellen zur Erzeugung von Wärme und Strom. (quashning regenerative s. 34ff.)

Einen weiteren Ansatzpunkt bietet der steigende globale Energiebedarf, welchem durch die Verbesserung der Energieeffizienz von Verbrauchernentegengewirkt werden kann. Die Einsparpotenziale bestehen bei privaten Hasuhalten, wie auch in der Industrie. Die Bundesregierung in Deutschland sieht darin außerdem einen Investitionsmotor, denn durch die Einsparungen entsteht mehr monbetärer Spielraum für Investitoinen und Konsum (Bilanz Energiewende Regierung S.2)

Damit ganzheitliche Lösungsansätze entwickelt und die Energiewende erfolgreich gemanagt werden kann, nimmt die Forschung eine zentrale Rolle ein. Diese umfasst das Erforschen neuer Technologien und das Optimieren und kontinuierliche Verbessern bestehender Technologien sowie die Verbesserung der Energieeffizienz. Die Bundesregierung maßt dabei Deutschalnd eine Vorreiterolle zu und fördert diese durch Forschungsprogramme zur Energieversorgung aus Erneuerbaren Energiequellen, der Energiepseuiceherung , und der Verbesserung der Energieeffizienz (Bilanz Energiewende Regierung S.11)
 
Einen Teil zu dieser Forschung und damit zur Vebesserung von solaren Anwendungen und deren Effizienz trägt die Hochschule Karlsruhe mit ihrem Forschungsprojekt bei, welches die Modellprädiktive Regelung einer Anlage zur solaren Klimatisierung des Fakultätsgebäude zum Ziel hat.

Die Modellprädiktive Regelung beschäftigt sich damit, wie ein allgemeiner Prozess oder ein technisches System, unter Zhilfenahme eines mathematischen Modells desselben, anhand eines gewählten Optimimalitätskriterium optimal -- im mathematisch exakten Sinne -- geregelt werden kann und trägt damit zur Verbesserung der Energieffizienz bei. Um  physikalisch motivierte Modelle der zur Regelung von Anlagen zu bilden, ist ein grundlegendes Verständnis der ablaufenden Prozesse unumgänglich wodurch weitere Verbesserungspotenizale einzelner Prozesse aufgedeckt werden können.

Die Anlage nutzt die in Solarkollektoren gewonnene Wärmeenergie, um eine Adsorptionskälteanlage anzutreiben. Da sich jedoch die Planung und Installation der Anlage aufgrund einiger Probleme verzögert hat, befindet sich diese derzeit noch im Aufbau. 
(https://www.hs-karlsruhe.de/fakultaeten/w/projekte.html)

Um bereits vorab nützliche Erfahrungen für die Inbetriebnahme der Solaranlage zu sammeln und erste Forschungsergebnisse  zu erzielen, soll diese durch eine kleine Anlage mit solarer Anwendung ergänzt werden. Die kleinen Anlage sollen komplementäre Eigeschaften zur großen Solaranlage besitzen und sich ebenfalls für die Modellprädiktive Regelung eignen.

Die Arbeit dient also dazu, Erfahrungen bei der Inbetriebnahme zu sammeln, um die Inbetriebnahme den Anlauf der solaren Klimatisierungsnanlage zu beschleunigen, und einen komplementären Forschungsbeitrag für die Modellprädiktive Regeluing von solaren Anwendungen zu leisten. 

\section{Zielsetzung und Aufbau der Arbeit}
\label{sec:ziel}

Das übergeordnete Ziel dieser Arbeit ist es also, die Forschung der Hochschule Karlsruhe auf dem Gebiet der Modellprädiktiven Regelung von solaren Anwendungen zu erweitern.
Konkret soll mit Hilfe einer kleiner Anlage mit solarer Anwendung jene Chancen nutzen, die die große Anlage nicht bieten kann und somit möglichst großen Nutzen zu realisieren. 

Als Problemstellung dieser Arbeit stellt sich damit die Forschungsfrage, wie eine Anlage zur Raumtemperaturregelung und ein mathematische Modell derselben aufgebaut sein müssen, um die Regelung einer Raumtemperatur mit Modellprädiktiver Regelung zu ermöglichen.


Aus der Forschungsfrage lässt sich das konkretes Ziel ableiten, eine entsprechende Anlage zu Konzipieren, zu Planen und Umzusetzen. Des Weiteren erfolgt die Bildung eines Modells für die Nutzung mit Modellprädiktiver Regelung.
Damit soll also eine Forschungsumgebung geschaffen werden, um verschiedene Steuerungs- und Regelungssystematiken zu untersuchen sowie eigene Regelungsalgorithmen zu entwickeln. Weiterhin soll mit dieder Arbeit Know-How generiert werden, dass bei der weiteren Forschung von Nutzen ist. Die konkreten Einsatzziele der Anlage wurden gemeinsam mit den beiden Forschungsprojektverangtwortlichen der Hochschule Karlsruhe, in Person von Herrn \textsc{Adrian Bürger} und \textsc{Markus Bohlayer}, definiert und sind in Kapitel \ref{sec:anforderungen} detailliert beschrieben.


%Here We go!!!! Nur noch verknüpfen welches kapitelk was und wie dem ziel dient. und der Forschungsfrage


Im Rahmen dieser Arbeit werden -- anschließend an die Einleitung in \ref{chap:einleitung} -- die theoretischen Grundlagen in \ref{chap:theoretischegrundlagen} ausgeführt. Zunächst wird die grundlegenden Theorie zu \acrlong{mpc} vorgestellt bevor anschließend weitere technische Grundlagen erklärt, welche für das weitere Verständnis dieser Arbeit benötigt werden. 

%Zu konkret evtl%Diese umfassen zunächst die thermodynamischen Grundlagen zur Modellbildung, die Beschreibung der Hardware- und Software-Schnittstellen des realisierten, technischen Systems.

Danach wird das technische System in Kapitel \ref{chap:anlagendesign} Schritt für Schritt entwickelt, ausgehend von der Idee und den räumlichen Gegebenheiten/Nebenbedingungen, und weiter konkretisiert bis zur realisierten Umsetzung in eine funktionierende Anlage.



Dementsprechend werden zunächst das Konzept, die Planung und die technische Umsetzung der konkreten Anlage dargestellt, bevor anschließend die theoretischen Grundlagen von \acrlong{mpc} und zur die Modellbildung erläutert werden. Anschließend wird das Modell für \acrlong{mpc} gebildet und ein erstes grobes Konzept zur Steuerung der Raumtemperatur vorgestellt. Abschließend wird eine Validierung des Modells versucht und findet eine Anpassung des Modells statt damit es künftig mit \acrlong{mpc} genutzt werden kann.