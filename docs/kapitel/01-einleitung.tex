%
% Einleitung
%
% @version 1.0
% @author wipatrick
% @created 22. November 2015
% @edited 

\setchapterpreamble[o]{%
\dictum[--- \textsc{Johann Wolfgang von Goethe}]{\Gun Erfolgreich zu sein setzt zwei Dinge voraus: Klare Ziele und den brennenden Wunsch, sie zu erreichen.\Gob}}
\renewcommand{\chapterheadstartvskip}{\vspace*{2cm}}

\chapter{Einleitung}
\label{chap:einleitung}
\pagenumbering{arabic}
\setcounter{page}{1}
\renewcommand{\chapterheadstartvskip}{\vspace*{-1cm}}


\section{Motivation und Problemstellung}
\label{sec:motivation}

Von allgemein solare Anwendung hin zu Forschung HSKA wird das Ziel sein.

Hinführung und Forschung der HSKA aufzeigen und daher einordnen Forschen derzeit bereits auf dem Gebiet (MPC mit) der solaren Anwendungen. Derzeit wird eine große Solaranlage zur Kühlung des Atriums installiert. Die Anlage nutzt die in Solarkollektoren gewonnene Wärmeenergie um eine Adsorptionskälteanlage anzutreiben. Die Anlage ist aufgrund ihrer Größe träge in der Reaktion und bisher noch im Aufbau. Komplementär dazu soll eine kleine Anlage mit solarem Bezug aufgebaut werden um die Forschung auch für ergänzende Dinge betreiben zu können.

Die Modellprädiktive Regelung beschäftigt sich damit, ein technisches System oder allgemeiner einen Prozess -- im mathematisch exakten Sinne -- optimal zu regeln.

Im Dialog mit den Projektverantwortlichen der Hochschule Karlsruhe für die Erforschung von solaren Anwendungen, in Person von Herrn \textsc{Adrian Bürger} und \textsc{Markus Bohlayer}, die Einsatzziele gemeinsam erarbeitet und sind im Detail in Kapitel \ref{sec:anforderungen} ausgeführt.

Wichtigste Ergebnisse aus Dialog ist die Motivation dieser Forschungsarbeit:

Die Motivation dieser Arbeit ist also, einen komplementären Forschungsbeitrag für solare Anwendungen zu leisten im Hinblick auf die große Anlage.

Dass heißt jene Chancen zu nutzen, die die große Anlage nicht bieten kann einen möglichst großen Nutzen zu realisieren, also was die große Anlage nicht leisten kann und die große Anlage zu entlasten im Sinne von Empfindlichkeit/Versuche.

Außerdem sollen möglichst Erfahrungen gesammelt werden, die beim Anlauf der großen Anlage von Nutzen sein können.


!!!!!!!!!!!!!!! Forschungsfrage, vorher Begriffe klären: MPC, Anlage, Modell, Steuerung/Regelung
Als Forschungsfrage und Problemstelltung dieser Arbeit ergibt sich die Frage, wie eine Anlage und ein Modell derselben aufgebaut sein müssen, um die Regelung einer Raumtemperatur mit Modellprädiktiver Regelung zu ermöglichen.
!!!!!!!!!!!!!!!

\section{Zielsetzung und Aufbau der Arbeit}
\label{sec:ziel}

Das übergeordnete Ziel dieser Arbeit ist es, eine Test- und Anwendungsumgebung zu schaffen, um die Forschung der Hochschule Karlsruhe auf dem Gebiet der Modellprädiktiven Regelung von solaren Anwendungen weitere Forschung weiter voran zu treiben und komplementär zu ergänzen.

Aus der Forschungsfrage lässt sich als konkretes Ziel der Arbeit ableiten, die Konzeption, Planung und Umsetzung einer Anlage und Bildung eines Modells zur Regelung einer Raumtemperatur mit Modellprädiktiver Regelung.
Also ein Labor
Weiterhin soll mit dieder Arbeit Know-How und Erfahrung generiert werden das bei der weiteren Forschung von Nutzen sein kann/ist.


Im Rahmen dieser Arbeit werden -- anschließend an die Einleitung in \ref{chap:einleitung} -- die theoretischen Grundlagen in \ref{chap:theoretischegrundlagen} ausgeführt. Zunächst wird die grundlegenden Theorie zu \acrlong{mpc} vorgestellt bevor anschließend weitere technische Grundlagen erklärt, welche für das weitere Verständnis dieser Arbeit benötigt werden. 

%Zu konkret evtl%Diese umfassen zunächst die thermodynamischen Grundlagen zur Modellbildung, die Beschreibung der Hardware- und Software-Schnittstellen des realisierten, technischen Systems.

Danach wird das technische System in Kapitel \ref{chap:anlagendesign} Schritt für Schritt entwickelt, ausgehend von der Idee und den räumlichen Gegebenheiten/Nebenbedingungen, und weiter konkretisiert bis zur realisierten Umsetzung in eine funktionierende Anlage.



Dementsprechend werden zunächst das Konzept, die Planung und die technische Umsetzung der konkreten Anlage dargestellt, bevor anschließend die theoretischen Grundlagen von \acrlong{mpc} und zur die Modellbildung erläutert werden. Anschließend wird das Modell für \acrlong{mpc} gebildet und ein erstes grobes Konzept zur Steuerung der Raumtemperatur vorgestellt. Abschließend wird eine Validierung des Modells versucht und findet eine Anpassung des Modells statt damit es künftig mit \acrlong{mpc} genutzt werden kann.