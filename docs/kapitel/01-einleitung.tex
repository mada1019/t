\setchapterpreamble[o]{%
\dictum[--- \textsc{Johann Wolfgang von Goethe}, \emph{deutscher Dichter und Schriftsteller}]{\Gun Erfolgreich zu sein setzt zwei Dinge voraus: Klare Ziele und den brennenden Wunsch, sie zu erreichen.\Gob}}
\renewcommand{\chapterheadstartvskip}{\vspace*{2cm}}

\chapter{Einleitung}
\label{chap:einleitung}
\pagenumbering{arabic}
\setcounter{page}{1}
\renewcommand{\chapterheadstartvskip}{\vspace*{-1cm}}


\section{Motivation und Problemstellung}
\label{sec:motivation}

Die neueren Entwicklungen in den südlichen und östlichen Staaten Asiens, allen voran China und Indien, haben einen starken Einfluss auf den enormen Anstieg des weltweiten, zukünftigen Energiebedarfs. In einem zentralen Szenario ihrer Prognosen, schätzt die \acrlong{iea} (\acrshort{iea}) den Energieverbrauch im Jahr 2040 um ein Drittel höher als im Jahr 2015 \cite[S.~1]{in15}.

Das \acrlong{ipcc} (\acrshort{ipcc}) betrachtet den Menschen als eine der Hauptursachen des Klimawandels, insbesondere jedoch als Verursacher für die globale Erderwärmung \cite[S.~V]{ip14}. Eine Schlüsselrolle bei der globalen Erwärmung spielt der Ausstoß von Treibhausgasen, wovon ein großer Anteil bei der Erzeugung von Elektrizitäts- und Wärmeenergie entsteht und nebenbei große Mengen an CO\textsubscript{2} in die Erdatmosphäre entweichen\cite[S.~47]{ip14}.

Die Bedeutung der Energieerzeugung wird daher auch in Zukunft immer weiter zunehmen. Da die Energieerzeugung aus Erneuerbaren Energien ohne den Ausstoß schädlicher Treibhausgase auskommen, bietet der Ausbau der erneuerbaren Energien  ein enorme Chance. Dies wird beispielsweise durch die Zielsetzung der Bundesregierung aus Deutschland belegt, die sich vorgenommen hat 80 \$ des Energiebedarfs im Jahre 2050 aus Erneuerbaren Energien zu decken \cite[S.~2]{bi15}.
Als Erneuerbare Energien werden Energiequellen bezeichnet, die aus Sicht des menschlichem Zeithorizonts unerschöpflich sind und daher ein gewaltiges Potenzial mit sich bringen. Die Erneuerbaren Energien umfassen die Planetenenergie durch Gravitation und geothermische Energie -- das jedoch mit Abstand größte Energieangebot bietet die Sonnenenergie. Das Angebot an Sonnenenergie übersteigt den gesamten weltweiten Energiebedarf um ein Vielfaches und könnte diesen daher theoretisch vollständig decken \cite[S.~34f.]{qu11}.

Bevor das Potenzial zur Deckung des Weltenergiebedarfs aus Regenerativen Energien genutzt werden kann, gilt es noch einige Problemstellungen zu lösen.
Eine davon ist die Frage, wie sich die technische Umwandlung der Solarenergie in eine nutzbare Energieform, wie beispielsweise Wärme oder Elektrizität, möglichst effizient realisieren lässt. Hiermit verbunden ist ebenfalls die Frage, wie bereits bestehende Technologien weiter optimiert werden können. Eine dieser Technologien, die von Bedeutung sind, sind solarthermische Kraftwerke, welche die Solarenergie zunächst in Wärme und anschließend in elektrische Energie umwandeln. Eine weitere derartige Technologie stellen die Solarkollektoren und -zellen dar, die zur CO\textsubscript{2}-freien Erzeugung von Wärme und Strom eingesetzt werden\cite[S.~36f.]{qu11}.

Ein zweites großes Problem ist der steigende globale Energiebedarf, welchem durch eine Erhöhung der Energieeffizienz entgegengewirkt werden kann. Die größten Einsparpotenziale sind dabei nicht in dem Bereich privater Haushalte zu finden, sondern in der Industrie. Die Bundesregierung von Deutschland sieht in den besagten Einsparpotenzialen einen Investitionsmotor. Werden diese realisiert, bestehe ein größerer monetärer Spielraum für Investitionen der Industrie und den Konsum der Privathaushalte \cite[S.~2]{bi15}.

Damit ganzheitliche Lösungsansätze entwickelt und die Energiewende erfolgreich gemanagt werden kann, ist die Forschung von zentraler Bedeutung. Sie umfasst die kontinuierliche Verbesserung bestehender und die Entwicklung neuer Technologien. Die Bundesregierung maßt Deutschland im Zuge der Energiewende eine Vorreiterrolle zu und versucht diese durch verschiedene Forschungsprogramme zu untermauern. Hierunter fallen Projekte zur Energieversorgung aus Erneuerbaren Energiequellen, der Energiespeicherung und der Verbesserung der Energieeffizienz \cite[S.~11]{bi15}.
 
Auch die Hochschule Karlsruhe trägt durch eines ihrer Forschungsprojekte einen Teil dazu bei, welches die \acrlong{mpr} (\acrshort{mpr})\footnote{Im Englischen und in der einschlägigen Literatur wird die \acrlong{mpr} auch als \acrlong{mpc} (\acrshort{mpc}) bezeichnet.} einer Anlage zur solaren Klimatisierung eines Fakultätsgebäudes zum Ziel hat.

Die \acrlong{mpr} beschäftigt sich damit, wie ein allgemeiner technischer Prozess anhand eines gewählten Optimalitätskriteriums, unter Zuhilfenahme eines mathematischen Modells desselben, optimal -- im mathematisch exakten Sinne -- geregelt werden kann. Wird der minimale Verbrauch von Energie als Optimalitätskriterium gewählt, insbesondere von nicht erneuerbar erzeugten Energien, trägt die \acrshort{mpr} zur Verbesserung der Energieeffizienz bei.
Bei der Bildung von komplexen und physikalisch-motivierten Modellen wird ein grundlegendes Verständnis für die einzelnen Bauteile und die ablaufenden Prozesse benötigt, wodurch mögliche Verbesserungspotenziale einzelner Prozesse und Komponenten aufgedeckt werden können.

Konkret umfasst das Forschungsprojekt der Hochschule Karlsruhe den Aufbau einer solaren Anlage. Sie nutzt die in Solarkollektoren gewonnene Wärmeenergie, um eine Adsorptionskälteanlage anzutreiben und dadurch für die Kühlung des K~Gebäudes zu sorgen. Die Installation der Anlage hat sich aufgrund einiger Probleme verzögert und befindet sich daher derzeit noch im Aufbau \cite{hska}.

Um bereits vorab nützliche Erfahrungen für die Inbetriebnahme der großen Solaranlage zu sammeln und erste Forschungsergebnisse zur \acrlong{mpr} zu erzielen, soll diese um eine kleine Anlage mit solarer Anwendung ergänzt werden. Die kleine Anlage soll komplementäre Eigenschaften zur solaren Klimatisierung besitzen und sich für eine \acrlong{mpr} eignen.

Diese Arbeit beschäftigt sich mit der Konzeption, Umsetzung und Inbetriebnahme einer kleinen Anlage, mit deren Hilfe nützliche Erfahrungen für die Inbetriebnahme der großen Solaranlage gesammelt werden sollen, um diese zu vereinfachen und deren Anlaufzeit zu verkürzen. Darüber hinaus soll durch diese Arbeit komplementärer Forschungsbeitrag zur großen Solaranlage im Bereich der \acrlong{mpr} realisiert werden.

\section{Zielsetzung und Aufbau der Arbeit}
\label{sec:ziel}

Das übergeordnete Ziel dieser Arbeit ist es, die Forschung der Hochschule Karlsruhe auf dem Gebiet der Modellprädiktiven Regelung von solaren Anwendungen voranzutreiben.
Konkret sollen durch die Konzeption, Planung und Installation einer kleinen Anlage mit solarer Anwendung jene Chancen genutzt werden, welche die große Anlage nicht bietet.
Als einfache solare Anwendung, die mit vergleichsweise wenig Aufwand realisierbar ist, bietet sich die Regelung der Temperatur innerhalb eines sonnenbestrahlten Raumes an.

Als Problemstellung dieser Arbeit ergibt sich damit die Forschungsfrage, wie eine Anlage zur Raumtemperaturregelung und ein mathematisches Modell derselben aufgebaut sein müssen, um eine \acrlong{mpr} der Raumtemperatur zu ermöglichen.

Aus jener Forschungsfrage lässt sich das konkrete Ziel ableiten, eine entsprechende Anlage zu Planen und zu Installieren. 
Zudem bedarf es der Bildung eines Modells für die Modellprädiktive Regelung der kleinen Anlage.
Es soll also eine Forschungsumgebung zur Untersuchung verschiedener Steuerungen und Regelungssystematiken sowie zur Entwicklung eigener Regelungsalgorithmen geschaffen werden.
Die Eigenschaften und Einsatzziele der Anlage wurden gemeinsam mit den beiden Projektverantwortlichen der Hochschule Karlsruhe, in Person von Herrn \textsc{Adrian Bürger} und \textsc{Markus Bohlayer}, festgelegt und sind in Kapitel \ref{sec:anforderungen} detailliert aufgeführt.


%Here We go!!!! Nur noch verknüpfen welches kapitel was und wie dem ziel dient.
Auf die Einleitung folgt eine Einführung in die theoretischen Grundlagen, die dem Verständnis beim Aufbau der Anlage und der Bildung des Modells dienen.

Um den Rahmen der Arbeit abzugrenzen, werden zunächst die Theorien der \acrlong{mpr} und der Optimalsteuerung erläutert, mit dem Ziel die Anforderungen an die Anlage und Modellbildung zu identifizieren.
Nach der Abgrenzung des Rahmens werden die allgemeinen Grundlagen zur Kommunikation von technischen Systemen thematisiert, welche bestimmte Grundlagen aus den Gebieten der Informatik, Elektro- und Nachrichtentechnik umfassen. Es wird ein theoretisches Fundament gelegt, auf welches im Anschluss die praktische Anwendung der Modbus Kommunikationstechnologie aufgebaut wird.
Der darauffolgende Abschnitt beschreibt die thermodynamischen Grundlagen, die bei der Beschreibung realer Prozesse in der Modellbildung in Kapitel \ref{chap:modellbildung} Anwendung finden.
Abschließend erfolgt eine allgemeine Einführung in die relevanten physikalischen und bautechnischen Grundlagen zur Solarstrahlung und Gebäudetechnik. Diese zielt darauf ab, ein Grundverständnis für den Einfluss der Solarstrahlung auf einen Raum zu entwickeln.

Das dritte Kapitel dient der Beschreibung des Aufbaus und der Funktionsweise der Anlage. Dazu werden zunächst die Anforderungen an die geplante Anlage definiert. Darauf basierend wird der Aufbau der Anlage von der Idee, bis hin zur abschließenden Beschreibung der installierten Anlage und deren Bedienung sukzessiv konkretisiert. Schließlich soll durch eine funktionierende Zweipunktregelung die Hypothese belegt werden, dass die Anlage fähig ist eine Raumtemperatur zu regeln.

Das Ziel des vierten Kapitels ist es, ein Modell der Anlage für die Modellprädiktive Regelung der Raumtemperatur zu entwickeln. Es wird ein einfaches Grundmodell aufgestellt, dessen Komplexität durch eine schrittweise Anpassung an die reale Umgebung stetig erhöht wird. Anschließend erfolgt die Simulation und Validierung des Modells, um eine Abschätzung für die Modellgüte zu erhalten. Im nachfolgenden Abschnitt erfolgt eine erneute Anpassung des Modells durch eine Parameterschätzung und abschließende Untersuchung der Modellgüte. Zuletzt wird die Eignung des Modells für den Einsatz mit \acrlong{mpr} untersucht.
  
Die Schlussbetrachtung soll eine Antwort auf die Forschungsfrage liefern, indem die Ergebnisse der Arbeit zusammengefasst werden. Zum Abschluss der Arbeit wird noch ein Ausblick gegeben, bei dem mögliche Ansatzpunkte für weitere Arbeiten aufgezeigt werden.