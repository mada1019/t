\setchapterpreamble[o]{%
\dictum[--- \textsc{Johann Wolfgang von Goethe}]{\Gun Erfolgreich zu sein setzt zwei Dinge voraus: Klare Ziele und den brennenden Wunsch, sie zu erreichen.\Gob}}
\renewcommand{\chapterheadstartvskip}{\vspace*{2cm}}

\chapter{Einleitung}
\label{chap:einleitung}
\pagenumbering{arabic}
\setcounter{page}{1}
\renewcommand{\chapterheadstartvskip}{\vspace*{-1cm}}


\section{Motivation und Problemstellung}
\label{sec:motivation}

Die neueren Entwicklungen in den südlichen und östlichen Staaten Asiens, allen voran China und Indien, haben einen starken Einfluss auf den enormen Anstieg des weltweiten, zukünftigen Energiebedarfs. In einem zentralen Szenario ihrer Prognosen, schätzt die \acrlong{iea} (\acrshort{iea}) den Energieverbrauch im Jahr 2040 um ein Drittel höher ein als im vergangenen Jahr 2015 \cite[S.~1]{in15}.

Zudem sieht das \acrlong{ipcc} (\acrshort{ipcc}) den Menschen als eine der Hauptursachen für den Klimawandel, insbesondere für die globale Erderwärmung \cite[S.~V]{ip14}. Eine Schlüsselrolle bei der globalen Erwärmung spielt der Ausstoss von Treibhausgasen. Einen großen Anteil davon entsteht bei der Erzeugung von Elektrizitäts- und Wärmeenergie, wobei große Mengen an CO\textsubscript{2} freigesetzt werden\cite[S.~47]{ip14}.

Daher wird die Bedeutung der Energieerzeugung auch in Zukunft weiter zunehmen. Der Energieerzeugung aus erneuerbaren Energien, welche ohne den Ausstoss schädlicher Treibhausgase auskommen, bietet sich damit ein enorme Chance. Dies wird beispielsweise durch die Zielsetzung der Bundesregierung in Deutschland belegt, deren Ziel es ist, im Jahre 2050 den Energiebedarf zu 80 \$ aus Erneuerbaren Energien zu decken \cite[S.~2]{bi15}.
Als Erneuerbare Energien werden Energiequellen bezeichnet, die nach menschlichem Zeithorizont unerschöpflich sind, worin sich ein gewaltiges Potenzial begründet. Sie umfassen die Planetenenergie durch Gravitation und geothermische Energie, das jedoch mit Abstand größte Energieangebot bietet die Sonnenenergie. Das Angebot an Sonnenenergie übersteigt den gesamten weltweiten Energiebedarf um ein vielfaches und könnte diesen daher theoretisch vollständig decken \cite[S.~34f.]{qu11}.

Bevor das Potenzial zur Deckung des Weltenergiebedarfs aus Regenerativen Energien genutzt werden kann, gilt es jedoch noch einige Probleme zu lösen.
Eines davon ist die Frage, wie die sich die technische Umwandlung der Solarenergie in eine nutzbare Energieform, wie beispielsweise Wärme oder Elektrizität, möglichst effizient realisieren lässt und bestehende Technologien weiter optimiert werden können. Wichtige Technologien bisher sind solarthermische Kraftwerke, welche die Solarenergie zunächst in Wärme und anschließend in elektrische Energie umwandeln, sowie Solarkollektoren und -zellen zur CO\textsubscript{2}-freien Erzeugung von Wärme und Strom \cite[S.~36f.]{qu11}.

Ein weiteres großes Problem ist der steigende globale Energiebedarf, welchem durch eine Erhöhung der Energieeffizienz entgegengewirkt werden kann. Die größten Einsparpotenziale bestehen dabei nicht in privaten Haushalten, sondern im Energieverbrauch der Industrie. Die Bundesregierung in Deutschland sieht darin zudem einen Investitionsmotor, da durch Einsparungen ein größerer monetärer Spielraum für Investitionen der Industrie und Konsum der Privathaushalte besteht \cite[S.~2]{bi15}.

Damit ganzheitliche Lösungsansätze entwickelt und die Energiewende erfolgreich gemanagt werden kann, ist die Forschung von zentraler Bedeutung. Die Forschung zur Energiewende umfasst die Entdeckung neuer Technologien sowie die kontinuierliche Verbesserung bestehender Technologien, unter anderem durch eine erhöhte Energieeffizienz. Die Bundesregierung maßt im Zuge der Energiewende Deutschland eine Vorreiterrolle zu und versucht diese durch verschiedene Forschungsprogramme zu untermauern. Sie umfassen Projekte zur Energieversorgung aus Erneuerbaren Energiequellen, der Energiespeicherung und der Verbesserung der Energieeffizienz \cite[S.~11]{bi15}.
 
Einen Teil zu dieser Forschung trägt die Hochschule Karlsruhe mit einem Forschungsprojekt bei, welches die \acrlong{mpr} (\acrshort{mpr})\footnote{Im Englischen und in der einschlägigen Literatur wird die \acrlong{mpr} auch als \acrlong{mpc} (\acrshort{mpc}) bezeichnet.} einer Anlage zur solaren Klimatisierung eines Fakultätsgebäude zum Ziel hat.

Die \acrlong{mpr} beschäftigt sich damit, wie allgemein ein Prozess, unter Zuhilfenahme eines mathematischen Modells desselben und anhand eines gewählten Optimalitätskriteriums, optimal -- im mathematisch exakten Sinne -- geregelt werden kann. Wird als Optimalitätskriterium der minimale Verbrauch von Energie gewählt, insbesondere von nicht-erneuerbar erzeugten Energien, trägt die \acrshort{mpr} damit zur Verbesserung der Energieeffizienz bei.
Bei der Bildung der benötigten, komplexen und physikalisch-motivierten Modelle ist außerdem ein grundlegendes Verständnis für die einzelnen Bauteile und die ablaufenden Prozesse erforderlich, wodurch Verbesserungspotenziale einzelner Prozesse und Komponenten aufgedeckt werden können.

Konkret umfasst das Forschungsprojekt der Hochschule Karlsruhe den Aufbau eine solaren Anlage, welche die in Solarkollektoren gewonnene Wärmeenergie nutzt, um eine Adsorptionskälteanlage anzutreiben und damit das K~Gebäude zu kühlen. Die Planung und Installation der Anlage hat sich aufgrund einiger Probleme verzögert und befindet sich daher derzeit noch im Aufbau \cite{hska}.

Um bereits vorab nützliche Erfahrungen für die Inbetriebnahme der großen Solaranlage zu sammeln sowie erste Forschungsergebnisse zur \acrlong{mpr} zu erzielen, soll diese durch eine kleine Anlage mit solarer Anwendung ergänzt werden. Diese kleine Anlage soll komplementäre Eigenschaften solaren Klimatisierung besitzen und sich ebenfalls für eine \acrlong{mpr} eignen.

Diese Arbeit übernimmt diese Aufgabe und versucht durch die Installation einer kleinen Anlage nützliche Erfahrungen zu sammeln, welche die Inbetriebnahme der großen Solaranlage vereinfachen und die Anlaufzeit verkürzen können. Weiterhin soll durch die Modellbildung für die \acrlong{mpr}, zusammen mit der kleinen Anlage, ein komplementärer Forschungsbeitrag zur großen Solaranlage realisiert werden.

\section{Zielsetzung und Aufbau der Arbeit}
\label{sec:ziel}

Das übergeordnete Ziel dieser Arbeit ist es also, die Forschung der Hochschule Karlsruhe auf dem Gebiet der Modellprädiktiven Regelung von solaren Anwendungen zu erweitern.
Konkret sollen durch die Konzeption, Planung und Installation einer kleinen Anlage mit solarer Anwendung jene Chancen genutzt werden, welche die große Anlage nicht bietet.
Als einfache solare Anwendung, die mit wenig Aufwand realisierbar ist, bietet sich die Regelung der Temperatur innerhalb eines sonnenbestrahlten Raumes an.

Als Problemstellung dieser Arbeit ergibt sich damit die Forschungsfrage, wie eine Anlage zur Raumtemperaturegelung und ein mathematisches Modell derselben aufgebaut sein müssen, um eine \acrlong{mpr} der Raumtemperatur zu ermöglichen.

Aus der Forschungsfrage lässt sich das konkretes Ziel ableiten, eine entsprechende Anlage zu Planen und zu Installieren. Weiterhin erfolgt die Bildung eines Modells für die Nutzung mit Modellprädiktiver Regelung.
Es soll also eine Forschungsumgebung geschaffen werden, um verschiedene Steuerungen und Regelungssystematiken zu untersuchen sowie eigene Regelungsalgorithmen entwickeln zu können.
Die Eigenschaften und Einsatzziele der Anlage wurden gemeinsam mit den beiden Projektverantwortlichen der Hochschule Karlsruhe, in Person von Herrn \textsc{Adrian Bürger} und \textsc{Markus Bohlayer}, definiert und sind in Kapitel \ref{sec:anforderungen} detailliert ausgeführt.


%Here We go!!!! Nur noch verknüpfen welches kapitelk was und wie dem ziel dient.
Auf die Einleitung folgt eine Einführung in die theoretischen Grundlagen, die dem Verständnis beim Aufbau der Anlage und der Bildung des Modells dienen.

Um den Rahmen der Arbeit abzugrenzen, werden zunächst die Theorien der \acrlong{mpr} und der Optimale Steuerung beschreiben, mit dem Ziel die Anforderungen an die Anlage und Modellbildung zu verstehen/definieren.
Nach der Abgrenzung des Rahmens werden die allgemeinen Grundlagen zur Kommunikation von technischen Systemen thematisiert, die Grundlagen aus der Informatik, Elektro- und Nachrichtentechnik zusammenfasst. Es wird ein theoretisches Fundament aufgebaut, dass im Anschluss daran mit der praktischen Anwendung der Modbus Kommunikationstechnologie gefüllt wird.
Der darauffolgende Abschnitt beschreibt die thermodynamischen Grundlagen, die zur Beschreibung der realen Prozesse bei der Modellbildung in Kapitel \ref{chap:modellbildung} Anwendung finden.
Abschließend erfolgt eine allgemeine Einführung in die physikalischen und bautechnischen Grundlagen zur Solarstrahlung und Gebäudetechnik, die darauf abzielen ein Grundverständnis für den Einfluss der Solarstrahlung auf einen Raum zu entwickeln.

Das dritte Kapitel dient der Beschreibung des Aufbaus und der Funktionsweise der Anlage. Dazu werden zunächst die Anforderungen an die geplante Anlage definiert. Darauf aufbauend wird der Anlagenaufbau schrittweise konkretisiert, von der Idee bis hin zur abschließenden Beschreibung der Funktionen und Bedienung der installierten Anlage. Abschließend wird eine erste funktionierende Zweipunktregelung vorgestellt und damit die Hypothese belegt, dass die Anlage fähig ist eine Raumtemperatur zu regeln.

Das Ziel des vierten Kapitels ist es, ein Modell der Anlage für die Modellprädiktive Regelung der Raumtemperatur zu entwickeln. Es wird ein einfaches Grundmodell aufgestellt, dessen Komplexität sukzessiv erhöht wird durch die Anpassung an die reale Anlage. Anschließend erfolgt die Simulation und Validierung des Modells, um eine Abschätzung für die Modellgüte zu erhalten. Der folgende Abschnitt dient einer Anpassung des Modells durch eine Parameterschätzung und der anschließenden Untersuchung der Modellgüte. Abschließend wird die Eignung des Modells für den Einsatz mit \acrlong{mpr} untersucht.
  
 Die Schlussbetrachtung soll eine Antwort auf die Forschungsfrage liefern indem die Ergebnisse der Arbeit zusammengefasst werden. Abschließend wird ein Ausblick gegeben und Ansatzpunkte für weitere Arbeiten aufgezeigt.