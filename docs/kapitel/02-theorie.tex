%
% Grundlagen

\chapter{Theoretische Grundlagen}
\label{chap:theoretischegrundlagen}

\section{Model Predictive Control}
\label{chap:mpc}


\acrlong{mpc} an sich ist eine Methodik zur Steuerung von Systemen. Diese versucht zunächst, zu sich periodisch wiederholenden, diskreten Zeitpunkten das Verhalten eines Systems in der Zukunft -- also einer immer gleich weit in die Zukunft hineinreichenden Periode -- zu beschreiben. Hierzu bedient \acrlong{mpc} sich der Kenntnis des aktuellen Zustandes und eines physikalischen-mathematischen Modells des Systems, um dessen zukünftiges Verhalten \Gun vorherzusagen \Gob bzw. abzubilden. Des Weiteren wird versucht das Verhalten des Systems mit minimalem Aufwand zu beeinflussen, um einem eigens- oder vordefinierten Zielkriterium zu folgen \acrlong{bzw} diesem zu entsprechen.

\begin{table}[H]
\centering
\small
\renewcommand{\arraystretch}{1.3}
\begin{threeparttable}
\begin{tabularx}{\textwidth}{p{0.018\textwidth}p{0.25\textwidth}C{0.1915\textwidth}C{0.1915\textwidth}C{0.1915\textwidth}}
\toprule

\multicolumn{1}{c}{N\textdegree}		&
Merkmal						&
\multicolumn{3}{c}{Merkmalsausprägung} \\

\cmidrule[0.4pt](r{0.25em}){1-1} 
\cmidrule[0.4pt](lr{0.25em}){2-2}
\cmidrule[0.4pt](lr{0.25em}){3-3}
\cmidrule[0.4pt](lr{0.25em}){4-4}
\cmidrule[0.4pt](l{0.25em}){5-5}

1						&
Bla					    &	
\ccol Bla	 	        &	
Bla			            & 	
Bla 					\\

2						&
Bla			            &	
\ccol Bla				&	
Bla					    &	
Bla 					\\

3						&
Bla			            &	
\ccol Bla		        &	
Bla				        &	
Bla 			        \\

4						&
Bla\tnote{1)}		    &
\multicolumn{2}{c}{\ccol Bla}   &
Bla 							\\

\bottomrule
\end{tabularx}
\begin{tablenotes}[]\footnotesize\singlespacing\setlength\labelsep{0pt}
\item[\textcolor{black!20}{\quadrat}] Der Fokus dieser Arbeit liegt auf ...
\item[1)] Die ist eine weitere Fußnote.
\end{tablenotes}
\end{threeparttable}
\caption[Tabellenunterschrift im Tabellenverzeichnis]{Tabellenunterschrift im Test.}
\label{tab:zsmmuster}
\end{table}



\section{Technische Grundlagen zu Bussystemen}
\label{sec:grundlagenbus}
In diesem Kapitel werden die Grundlagen von Hard- und Software beleuchtet die für die Kommunikation der Steuerung mit den einzelnen Anlagenteilen benötigt werden.
Diese umfassen zunächst Bussysteme im Allgemeinen und werden anhand des spezifischen/konkreten Anwendungsfalls Modbus erläutert.
Die Einführung wird sich an die Strukutr nach \cite{schn06} anlehnen.

\subsection{Bussysteme} 
Um Prozesse überwachen und steuern zu können müssen unter den einzelnen Einheiten innerhalb eines Systems Informationen ausgetauscht werden. Verknüpft man diese über Verbindungsleitungen, über die Informationen übertragen werden können, miteinander entstehen dabei Netzwerke.

Im Folgenden werden deshalb die Netzwerk-Grundlagen die im Rahmen dieser Arbeit von Relevanz sind dargestellt und erläutert.

Zunächst wird das Netzwerk anhand der geometrischen Anordnung der Verbindungsleitungen klassifiziert \cite[S.~1f.]{schn06}. 
Die einfachste Art ist zwei Kommunikationspartner direkt miteinander zu einer Zweipunktverbindung zu verbinden. Dazu werden jedoch bei einer großen ANzahl von Kommunikationspartnern sehr viele Verbindungsleitungen benötigt, was mit einer großen Anzahl von Schnittstellen und hohen Kosten einhergeht. Jedoch würden Fehler nicht das ganze Netzwerk sondern nur eine Verbindung zwischen zwei Partnern betreffen und der Fhler wäre leicht zu diagnostizieren.

(Um Zweipunktverbindung mit weniger Leitungen zu realisieren gibt es hzwei möglihckiten um die Kommuniktaion von mehreren Partnern über eine leitung zu ermöglichen, d.h. eine Signalbeeinflussung von Signalen unterineinander zu vermeiden : Zeitmultiplex und Frequenzmultiplex.) Überflüssig?!?!




\subsubsection{Elektrisches EIA-485 Netzwerk/Interface}
Hier wird EIA / RS485 dargestellt und abgegrenzt zu RS 232 und RS 422

\subsubsection{Hardware}
Kabel, Belegung



\subsection{Modbus}



Hier wird das Modbusprotokoll



\section{Technische Grundlagen zur Modellbildung}
\label{sec:grundlagenmodell}
In diesem Kapitel werden die technischen Grundlagen zur Bilanzierung, welche für die Modellbildung benötigt werden, erläutert.

\subsection{Thermodynamische Systeme}
Im Raummodell müssen Energieströme, genauer betrachtet Wärmeströme, untersucht werden. Um diese mit Hilfe von Bilanzierungen zu beschreiben folgt zunächst ein kurze Einführung in die Thermodynamische Systembildung nach \cite[S.~11ff.]{ba12}.


Die thermodynamischen Systeme sind dadurch charakterisiert, dass sie durch den zu untersuchenden Raum abgegrenzt sind. Sie dienen dem Zweck der Bilanzierung von Massen- und Energieströmen. Alles was diesen abgegrenzten Raum an den Systemgrenzen umgibt wird als Umgebung bezeichnet. Die begrenzenden Flächen können gedanklicher, physischer oder beider Natur sein, sie müssen lediglich eindeutig festgelegt sein.
Anhand der Eigenschaften der Systemgrenzen lassen sich die Systeme weiter differenzieren.
Bei Systemen deren Grenzen undurchlässig für Materie sind spricht man von geschlossenen Systemen. Die Grenzen eines solchen Systems sind meistens räumlich fest/definiert, müssen aber nicht gezwungenermaßen räumlich fest/fix sondern könne auch beweglich sein und werden dann durch das Volumen der Stoffmenge festgelegt.
Sind die Grenzen für Materie durchlässig spricht man von offenen Systemen die in der Regel durch räumlich festgelegte räumliche Grenzen begrenzt sind. Dieser wird auch als Kontrollraum zbw. Kontrollvolumen bezeichnet. Diese werden können von Stoffströmen durchflossen werden.
Ein abgeschlossenes System umfasst in der Regel mehrere Systeme \acrlong{bzw} ein System und dessen Umgebung sodass es zwischen dessen Grenzen und der Umgebung des abgeschlossenen Systems keine Wechselwirkungen gibt, dass heißt über dessen Grenzen hinweg fließen keine bzw. keine relevanten, dh kaum messbare, Flüsse von Materie und Energie.


Solche thermodynamischen Systeme werden durch physikalische Größen beschrieben, welche gleichzeitig seine Eigenschaften kennzeichnen. Im Rahmen dieser Arbeit ist es ausreichend die Vorgänge und Effekte auf Makroskopischer Ebene zu betrachten, daher lässt sich ein solches Systems mit wenigen Variablen/Größen beschreiben.
Deshalb wird innerhalb der Grenzen eines thermodynamischen Systems, also auch implizit für das Raummodell(Diese Annahme ist noch zu überprüfen und zu diskutieren), angenommen werden, dass die physikalischen Eigenschaften wie z.B. Temperatur, Druck und die chemische Zusammensetzung homogen ist, also an jeder Stelle die gleiche Ausprägung besitzt \cite[S.15]{ba12} . 
Die Variablen lassen sich in äußere, welche den mechanischen Zustand beschreiben (Koordinaten im Raum, relative Geschw. zum Beobachter), und innere, welche den thermodynamischen Zustand der MAterie innerhalb der Systemgrenzen beschreiben , Größen aufteilen.
Der Zustand eines Systems wird also durch die Variablen die ihn beschreiben charakterisiert/definiert, welshalb die Variablen, wie auch im Folgenden, als Zustandsgrößen bezeichnet werden.
\cite[S.13~f.]{ba12}


Da wir im Rahmen von \acrlong{mpc} Zustände und deren Änderungen untersuchen müssen auch Zustandsänderungen untersucht werden. Zustandsänderungen eines Systems werden durch Änderungen (also Flüsse) von Energie und Materie/Masse über dessen Grenzen hinweg bedingt. Diese finden meist im Austausch der Umgebung statt. Während einer solchen Änderung des Systemzustands wird ein Prozess durchlaufen, der eine zeitliche Abfolge von Ereignissen ist. Eine gleiche Zustandsänderung kann also durch verschiedene Prozesse bewirkt werden. Ein Prozess beschreibt also mehr als die reine Zustandsänderung des Systems sondern viel mehr die Beziehungen zwischen einem System und seiner Umgebung. 
Ein Prozess ist also umfassender als eine Zustandsänderung und diese weist lediglich auf einen ablaufenden Prozess hin.
Ein Prozess kann aber auch innerhalb eines Systems ohne äußere Einwirkungen stattfinden, z.B. durch Aufheben innnerer Hemmungen oder Zwängen von Außen. Dies sind meist von selbst statfindende Prozesse in abgeschlossenen Systemen, wie auch relevant für die Modellbildung in Kapitel \ref{chap:modellbildung}. Diese Prozesse werden Ausgleichsprozesse genannt, die danach streben einen Gleichgewichtszustand als Endzustand einzunehmen(\Gun Erfahrungsatz\Gob dass ein sich selbst überlassenes abgeschl. System einem GG-Zustand zustrebt). Solche Ausgleichprozesse repräsentieren Wechselwirkungen zwischen verschiedenen Teilen des abgeschlossenen Systems, die dabei versuchen die Zustandsgrößen auszugleichen und in den GG-Zustand zu kommen, in dem das System vsich von sich aus nicht ändert sondern nur durch äußere Eingriffe, da das System den GG nie von selbst verlässt. Ein abgeschlossenes System strebt also imme einem GG-Zustand hinterher, der durch die Zustände in den einzelnen Teilen(Subsystemen) bestimmt wird. 
\cite[S.21~f.]{ba12}

Reversible Prozesse, da angenommen dass Umgebung groß genug damit die \Gun kleinen\Gob Energieströme vernachlässigt werden können. -notwendig?!


\subsection{Erster Hauptsatz der Thermodynamik}

\cite[S.~43ff.]{ba12}
Der erste Hauptsatz der Thermodynamik erweitert den mechanischen Ernergieerhaltungssatz um die Energieformen Wärme und Energie und handelt ganz allgemein vom Prinzip der Energieerhaltung und dient der Bilanzierung von Systemen und wird später bei der Modellbildung des Raumes Anwendung finden. 
Die Energie eines Systems $E$ besteht aus potenzieller $E_{pot}$ und kinetischer Energie $E_{kin}$ wie in der Mechanik und wird durch die Innere Energie $U$ ergänzt wie in nachstehender Gleichung beschrieben \cite[S.~49]{ba12}

\begin{equation}
\label{eq:energie}
E := E_{pot} + E_{kin} + U
\end{equation}
 
Die mechanischen Energie kann bei den nachfogenden Betrachtungen vernachlässigt werden, da es sich um ein ortsfestes System handelt, das heißt die potentielle Energie ist konstant, in das und aus dem heraus keine Ströme fließen und es sich eslbts nicht bewegt, das heißt die kinetsiche Energie bleobt evenfalls kosntant und können somit beide vernachlässigt werden und die Energie des Systems wird verinfacht nur durch durch die innere Energie angenommen.



Die innere Energie ist abhängig von der spezifischen Wärmekapazität und der Masse innerhalb eines Systems und berechnet sich nach der veränderten Gleichung aus \cite[S.~54]{ba12} und entspricht der Energie der Masse eines Systems:

\begin{equation}
\label{eq:innereenergie}
U := m*c_p*T bzw. m*c_p*(t-T_0)+u_0
\end{equation}

Erst qualitativ, dann quantitaiv
Der 1. Hauptsatz besagt, dass jedes System eine Energie als Zustandsgröße besitzt die sich aus den folgenden zusammensetzt:
mechanisch
wärme
innere Energie

\cite[S.~48]{ba12}
Außerdem kann sich energie nur durch transport über die grenzen hinweg ändern und es gilt das prinzip der nergieerhaltung. Möglich trabsportformen, Verricgten von arbeit --> wie mechanik, d.h. umewandlung von enegrie
Übergang von wärme, das heiosst qärmetrabsprot, und Transport von Materie, da Mateire gleiche energie, also auch durch Stoffstransport.
Daraus kfolgt die Energiebilanzgleichung um den effekt quanititiv zu bescheibeen
Energie System Gesamt E= Ekin+Epot+U --Y Auch System in Ruhe, d.h. ekin=epot=0 hat Energie
U ist innere Energie und ebschreibt die Energie die innerhalb eines Systems, durch bestandteile und ist abhönigg von der der masse, da Masse=energie
Energie eines systems ist konstant und berechnet sich nach folgender Gleichung \cite[S.~54]{ba12}

\begin{equation}
\label{eq:hauptsatz}
Q_{12} + W_{12} = E_2 - E_1
\end{equation}


\subsection{Wärmeübertragung}
Da zur Betsimmung und Steuerung einer Raumheizungsnalage die Betrachtung von Wärmeströmen unumgänglich ist werden die Grundlagen dazu im Folgenden erläutert.
Die Wärmeübertragung kann grundsätzlich nur durch zwei Arten stattfinden, durch Strahlung, die ohne stofflichen Träger durch elektromagnetische Wellen erfolgt, welche keine weitere Relevanz für weitere Betrachtungen hat und deshalb hier nicht erläutert wird, und durch die Wärmeleitung, die sich wiederum in die Leitung und die Konvektion aufteilt. \cite[S.~3f.]{bo14}

Die Wärmeübertragung wird druch den Wärmestrom $\dot{Q}$ beschrieben, der quantifiziert wieviel Wärme pro Zeiteinheit $[W]$ übertragen wird \cite[S.~5]{bo14}.
Für die weitere Betrachtung im Rahmen dieser Arbeit ist lediglich die Wärmeleitung ohne KOnvektion von Relevanz, weshalb diese nun näher erläutert wird.
Kinetische Kopplungsgleichung\cite[S.~7]{bo14}
Gleichung Qdot =k*A*deltaT
Geben Wärmestrom als Funktion an von verschiedenen Paramatern, hier Fläche an der Austausch stattfindet, Temperaturdifferenz und Wärmedurchgangszahl bzw Wärmeübergangszahl an.
Wärmedurchgangskoeefizient oder auch U Wert
Übertsragung folgt ab Seite 17 Wärmebuch
